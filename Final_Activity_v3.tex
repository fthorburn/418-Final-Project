% Options for packages loaded elsewhere
\PassOptionsToPackage{unicode}{hyperref}
\PassOptionsToPackage{hyphens}{url}
%
\documentclass[
]{article}
\usepackage{amsmath,amssymb}
\usepackage{lmodern}
\usepackage{iftex}
\ifPDFTeX
  \usepackage[T1]{fontenc}
  \usepackage[utf8]{inputenc}
  \usepackage{textcomp} % provide euro and other symbols
\else % if luatex or xetex
  \usepackage{unicode-math}
  \defaultfontfeatures{Scale=MatchLowercase}
  \defaultfontfeatures[\rmfamily]{Ligatures=TeX,Scale=1}
\fi
% Use upquote if available, for straight quotes in verbatim environments
\IfFileExists{upquote.sty}{\usepackage{upquote}}{}
\IfFileExists{microtype.sty}{% use microtype if available
  \usepackage[]{microtype}
  \UseMicrotypeSet[protrusion]{basicmath} % disable protrusion for tt fonts
}{}
\makeatletter
\@ifundefined{KOMAClassName}{% if non-KOMA class
  \IfFileExists{parskip.sty}{%
    \usepackage{parskip}
  }{% else
    \setlength{\parindent}{0pt}
    \setlength{\parskip}{6pt plus 2pt minus 1pt}}
}{% if KOMA class
  \KOMAoptions{parskip=half}}
\makeatother
\usepackage{xcolor}
\usepackage[margin=1in]{geometry}
\usepackage{color}
\usepackage{fancyvrb}
\newcommand{\VerbBar}{|}
\newcommand{\VERB}{\Verb[commandchars=\\\{\}]}
\DefineVerbatimEnvironment{Highlighting}{Verbatim}{commandchars=\\\{\}}
% Add ',fontsize=\small' for more characters per line
\usepackage{framed}
\definecolor{shadecolor}{RGB}{248,248,248}
\newenvironment{Shaded}{\begin{snugshade}}{\end{snugshade}}
\newcommand{\AlertTok}[1]{\textcolor[rgb]{0.94,0.16,0.16}{#1}}
\newcommand{\AnnotationTok}[1]{\textcolor[rgb]{0.56,0.35,0.01}{\textbf{\textit{#1}}}}
\newcommand{\AttributeTok}[1]{\textcolor[rgb]{0.77,0.63,0.00}{#1}}
\newcommand{\BaseNTok}[1]{\textcolor[rgb]{0.00,0.00,0.81}{#1}}
\newcommand{\BuiltInTok}[1]{#1}
\newcommand{\CharTok}[1]{\textcolor[rgb]{0.31,0.60,0.02}{#1}}
\newcommand{\CommentTok}[1]{\textcolor[rgb]{0.56,0.35,0.01}{\textit{#1}}}
\newcommand{\CommentVarTok}[1]{\textcolor[rgb]{0.56,0.35,0.01}{\textbf{\textit{#1}}}}
\newcommand{\ConstantTok}[1]{\textcolor[rgb]{0.00,0.00,0.00}{#1}}
\newcommand{\ControlFlowTok}[1]{\textcolor[rgb]{0.13,0.29,0.53}{\textbf{#1}}}
\newcommand{\DataTypeTok}[1]{\textcolor[rgb]{0.13,0.29,0.53}{#1}}
\newcommand{\DecValTok}[1]{\textcolor[rgb]{0.00,0.00,0.81}{#1}}
\newcommand{\DocumentationTok}[1]{\textcolor[rgb]{0.56,0.35,0.01}{\textbf{\textit{#1}}}}
\newcommand{\ErrorTok}[1]{\textcolor[rgb]{0.64,0.00,0.00}{\textbf{#1}}}
\newcommand{\ExtensionTok}[1]{#1}
\newcommand{\FloatTok}[1]{\textcolor[rgb]{0.00,0.00,0.81}{#1}}
\newcommand{\FunctionTok}[1]{\textcolor[rgb]{0.00,0.00,0.00}{#1}}
\newcommand{\ImportTok}[1]{#1}
\newcommand{\InformationTok}[1]{\textcolor[rgb]{0.56,0.35,0.01}{\textbf{\textit{#1}}}}
\newcommand{\KeywordTok}[1]{\textcolor[rgb]{0.13,0.29,0.53}{\textbf{#1}}}
\newcommand{\NormalTok}[1]{#1}
\newcommand{\OperatorTok}[1]{\textcolor[rgb]{0.81,0.36,0.00}{\textbf{#1}}}
\newcommand{\OtherTok}[1]{\textcolor[rgb]{0.56,0.35,0.01}{#1}}
\newcommand{\PreprocessorTok}[1]{\textcolor[rgb]{0.56,0.35,0.01}{\textit{#1}}}
\newcommand{\RegionMarkerTok}[1]{#1}
\newcommand{\SpecialCharTok}[1]{\textcolor[rgb]{0.00,0.00,0.00}{#1}}
\newcommand{\SpecialStringTok}[1]{\textcolor[rgb]{0.31,0.60,0.02}{#1}}
\newcommand{\StringTok}[1]{\textcolor[rgb]{0.31,0.60,0.02}{#1}}
\newcommand{\VariableTok}[1]{\textcolor[rgb]{0.00,0.00,0.00}{#1}}
\newcommand{\VerbatimStringTok}[1]{\textcolor[rgb]{0.31,0.60,0.02}{#1}}
\newcommand{\WarningTok}[1]{\textcolor[rgb]{0.56,0.35,0.01}{\textbf{\textit{#1}}}}
\usepackage{longtable,booktabs,array}
\usepackage{calc} % for calculating minipage widths
% Correct order of tables after \paragraph or \subparagraph
\usepackage{etoolbox}
\makeatletter
\patchcmd\longtable{\par}{\if@noskipsec\mbox{}\fi\par}{}{}
\makeatother
% Allow footnotes in longtable head/foot
\IfFileExists{footnotehyper.sty}{\usepackage{footnotehyper}}{\usepackage{footnote}}
\makesavenoteenv{longtable}
\usepackage{graphicx}
\makeatletter
\def\maxwidth{\ifdim\Gin@nat@width>\linewidth\linewidth\else\Gin@nat@width\fi}
\def\maxheight{\ifdim\Gin@nat@height>\textheight\textheight\else\Gin@nat@height\fi}
\makeatother
% Scale images if necessary, so that they will not overflow the page
% margins by default, and it is still possible to overwrite the defaults
% using explicit options in \includegraphics[width, height, ...]{}
\setkeys{Gin}{width=\maxwidth,height=\maxheight,keepaspectratio}
% Set default figure placement to htbp
\makeatletter
\def\fps@figure{htbp}
\makeatother
\setlength{\emergencystretch}{3em} % prevent overfull lines
\providecommand{\tightlist}{%
  \setlength{\itemsep}{0pt}\setlength{\parskip}{0pt}}
\setcounter{secnumdepth}{-\maxdimen} % remove section numbering
\ifLuaTeX
  \usepackage{selnolig}  % disable illegal ligatures
\fi
\IfFileExists{bookmark.sty}{\usepackage{bookmark}}{\usepackage{hyperref}}
\IfFileExists{xurl.sty}{\usepackage{xurl}}{} % add URL line breaks if available
\urlstyle{same} % disable monospaced font for URLs
\hypersetup{
  pdftitle={Final\_Activity\_v3},
  pdfauthor={Foster Thorburn},
  hidelinks,
  pdfcreator={LaTeX via pandoc}}

\title{Final\_Activity\_v3}
\author{Foster Thorburn}
\date{2022-11-30}

\begin{document}
\maketitle

\begin{Shaded}
\begin{Highlighting}[]
\CommentTok{\#edit test}
\FunctionTok{library}\NormalTok{(ggplot2)}
\FunctionTok{library}\NormalTok{(dplyr)}
\end{Highlighting}
\end{Shaded}

\begin{verbatim}
## 
## Attaching package: 'dplyr'
\end{verbatim}

\begin{verbatim}
## The following objects are masked from 'package:stats':
## 
##     filter, lag
\end{verbatim}

\begin{verbatim}
## The following objects are masked from 'package:base':
## 
##     intersect, setdiff, setequal, union
\end{verbatim}

\begin{Shaded}
\begin{Highlighting}[]
\FunctionTok{library}\NormalTok{(RWDataPlyr)}
\FunctionTok{library}\NormalTok{(readxl)}
\FunctionTok{library}\NormalTok{(skimr)}
\FunctionTok{library}\NormalTok{(forcats)}
\FunctionTok{library}\NormalTok{(janitor)}
\end{Highlighting}
\end{Shaded}

\begin{verbatim}
## 
## Attaching package: 'janitor'
\end{verbatim}

\begin{verbatim}
## The following objects are masked from 'package:stats':
## 
##     chisq.test, fisher.test
\end{verbatim}

\begin{Shaded}
\begin{Highlighting}[]
\FunctionTok{library}\NormalTok{(stringr)}
\FunctionTok{library}\NormalTok{(tidyr)}
\FunctionTok{library}\NormalTok{(ggthemes)}
\FunctionTok{library}\NormalTok{(tibble)}
\FunctionTok{library}\NormalTok{(purrr)}
\FunctionTok{library}\NormalTok{(ggplot2)}
\end{Highlighting}
\end{Shaded}

\hypertarget{reading-in-current-population-survey-december-2021-food-security-supplement}{%
\subsection{Reading in Current Population Survey, December 2021: Food
Security
Supplement}\label{reading-in-current-population-survey-december-2021-food-security-supplement}}

\begin{Shaded}
\begin{Highlighting}[]
\CommentTok{\# Read in csv file}
\NormalTok{dec21pub }\OtherTok{\textless{}{-}} \FunctionTok{read.csv}\NormalTok{(}\StringTok{"C:/Users/foste/Desktop/418 final project folder/dec21pub.csv"}\NormalTok{)}
\end{Highlighting}
\end{Shaded}

Subsetting Food Supplement data set for variables

\begin{Shaded}
\begin{Highlighting}[]
\NormalTok{data\_dec21 }\OtherTok{\textless{}{-}} \FunctionTok{select}\NormalTok{(dec21pub, GTCO, PRTAGE, PTDTRACE, PESEX, }
\NormalTok{                     PEMARITL, GESTFIPS, PEEDUCA,PEHSPNON, HRFS12M4, HRFS12M1, HRFS12MD)}

\CommentTok{\#removing old F.S data sets}
\FunctionTok{rm}\NormalTok{(dec21pub)}
\end{Highlighting}
\end{Shaded}

Creating unique County FIPS codes

\begin{Shaded}
\begin{Highlighting}[]
                            \CommentTok{\#state    \#county}
\CommentTok{\# Unique County FIPS code = [GESTFIPS][GTCO]}

\CommentTok{\#SATE{-}CODE}

\CommentTok{\#test var holds the value of the length of the state code}
\NormalTok{data\_dec21 }\OtherTok{\textless{}{-}}\NormalTok{ data\_dec21 }\SpecialCharTok{\%\textgreater{}\%} 
          \FunctionTok{mutate}\NormalTok{(}\AttributeTok{test\_var =} \FunctionTok{nchar}\NormalTok{(}\FunctionTok{as.character}\NormalTok{(}\FunctionTok{factor}\NormalTok{(GESTFIPS))))}

\CommentTok{\#code to add one zero to the front of the state code if there is only one value, if there is two adds nothing}
\NormalTok{data\_dec21 }\OtherTok{\textless{}{-}}\NormalTok{ data\_dec21 }\SpecialCharTok{\%\textgreater{}\%} 
  \FunctionTok{mutate}\NormalTok{(}\AttributeTok{test\_var2 =} \FunctionTok{ifelse}\NormalTok{(data\_dec21}\SpecialCharTok{$}\NormalTok{test\_var }\SpecialCharTok{==} \DecValTok{1}\NormalTok{, }\FunctionTok{str\_c}\NormalTok{(}\StringTok{"0"}\NormalTok{,data\_dec21}\SpecialCharTok{$}\NormalTok{GESTFIPS,}\AttributeTok{sep=}\StringTok{""}\NormalTok{),data\_dec21}\SpecialCharTok{$}\NormalTok{GESTFIPS))}


\CommentTok{\#COUNTY{-}CODE}

\CommentTok{\#test\_var\_county holds the value of the length of the county code}
\NormalTok{data\_dec21 }\OtherTok{\textless{}{-}}\NormalTok{ data\_dec21 }\SpecialCharTok{\%\textgreater{}\%} 
          \FunctionTok{mutate}\NormalTok{(}\AttributeTok{test\_var\_county =} \FunctionTok{nchar}\NormalTok{(}\FunctionTok{as.character}\NormalTok{(}\FunctionTok{factor}\NormalTok{(GTCO))))}


\CommentTok{\#code to add one zero to the front of the county code if there is less than or equal to two values }
\NormalTok{  data\_dec21 }\OtherTok{\textless{}{-}}\NormalTok{ data\_dec21 }\SpecialCharTok{\%\textgreater{}\%} 
  \FunctionTok{mutate}\NormalTok{(}\AttributeTok{test\_var2\_county =} \FunctionTok{ifelse}\NormalTok{(data\_dec21}\SpecialCharTok{$}\NormalTok{test\_var\_county }\SpecialCharTok{\textless{}=} \DecValTok{2}\NormalTok{, }\FunctionTok{str\_c}\NormalTok{(}\StringTok{"0"}\NormalTok{,data\_dec21}\SpecialCharTok{$}\NormalTok{GTCO,}\AttributeTok{sep=}\StringTok{""}\NormalTok{),data\_dec21}\SpecialCharTok{$}\NormalTok{GTCO))}

\CommentTok{\#Reruning the counter variable to find the length of the updated county code}
\NormalTok{data\_dec21 }\OtherTok{\textless{}{-}}\NormalTok{ data\_dec21 }\SpecialCharTok{\%\textgreater{}\%} 
          \FunctionTok{mutate}\NormalTok{(}\AttributeTok{test\_var\_county =} \FunctionTok{nchar}\NormalTok{(test\_var2\_county))}

\CommentTok{\#Iterating a second time to add another zero if there\textquotesingle{}s only two values, otherwise leave value be  }
\NormalTok{data\_dec21 }\OtherTok{\textless{}{-}}\NormalTok{ data\_dec21 }\SpecialCharTok{\%\textgreater{}\%} 
  \FunctionTok{mutate}\NormalTok{(}\AttributeTok{test\_var2\_county =} \FunctionTok{ifelse}\NormalTok{(data\_dec21}\SpecialCharTok{$}\NormalTok{test\_var\_county }\SpecialCharTok{\textless{}=} \DecValTok{2}\NormalTok{, }\FunctionTok{str\_c}\NormalTok{(}\StringTok{"0"}\NormalTok{,data\_dec21}\SpecialCharTok{$}\NormalTok{test\_var2\_county,}\AttributeTok{sep=}\StringTok{""}\NormalTok{),data\_dec21}\SpecialCharTok{$}\NormalTok{test\_var2\_county))}

\CommentTok{\#UNIQUE{-}COUNTY CODE}

\NormalTok{data\_dec21 }\OtherTok{\textless{}{-}}\NormalTok{ data\_dec21 }\SpecialCharTok{\%\textgreater{}\%} 
          \FunctionTok{mutate}\NormalTok{(}\AttributeTok{FIPS\_code =} \FunctionTok{str\_c}\NormalTok{(test\_var2,test\_var2\_county,}\AttributeTok{sep=}\StringTok{""}\NormalTok{))}



\CommentTok{\#example function}
\CommentTok{\#my\_funct \textless{}{-} function() \{}
\CommentTok{\#  print("test")}
\CommentTok{\#\}}
\end{Highlighting}
\end{Shaded}

Re-coding Demographic Variables

\begin{Shaded}
\begin{Highlighting}[]
\CommentTok{\#Replacing {-}1 values with NA for entire data set}
\NormalTok{data\_dec21 }\OtherTok{\textless{}{-}}\NormalTok{ data\_dec21 }\SpecialCharTok{\%\textgreater{}\%}
                  \FunctionTok{mutate}\NormalTok{(}\FunctionTok{across}\NormalTok{(}\FunctionTok{everything}\NormalTok{(), }\SpecialCharTok{\textasciitilde{}}\FunctionTok{na\_if}\NormalTok{(}\FunctionTok{str\_trim}\NormalTok{(.x), }\StringTok{"{-}1"}\NormalTok{)))}
\CommentTok{\#Replacing {-}9 values with NA for entire data set}
\NormalTok{data\_dec21 }\OtherTok{\textless{}{-}}\NormalTok{ data\_dec21 }\SpecialCharTok{\%\textgreater{}\%}
                  \FunctionTok{mutate}\NormalTok{(}\FunctionTok{across}\NormalTok{(}\FunctionTok{everything}\NormalTok{(), }\SpecialCharTok{\textasciitilde{}}\FunctionTok{na\_if}\NormalTok{(}\FunctionTok{str\_trim}\NormalTok{(.x), }\StringTok{"{-}9"}\NormalTok{)))}
\CommentTok{\#Replacing {-}6 values with NA for entire data set}
\NormalTok{data\_dec21 }\OtherTok{\textless{}{-}}\NormalTok{ data\_dec21 }\SpecialCharTok{\%\textgreater{}\%}
                  \FunctionTok{mutate}\NormalTok{(}\FunctionTok{across}\NormalTok{(}\FunctionTok{everything}\NormalTok{(), }\SpecialCharTok{\textasciitilde{}}\FunctionTok{na\_if}\NormalTok{(}\FunctionTok{str\_trim}\NormalTok{(.x), }\StringTok{"{-}6"}\NormalTok{)))}

\CommentTok{\#Recoding Hispanic Variable {-} PEHSPNON}
\NormalTok{data\_dec21 }\OtherTok{\textless{}{-}}\NormalTok{ data\_dec21 }\SpecialCharTok{\%\textgreater{}\%} 
  \FunctionTok{mutate}\NormalTok{(}\AttributeTok{Hispanic =} \FunctionTok{factor}\NormalTok{(PEHSPNON) }\SpecialCharTok{\%\textgreater{}\%} 
    \FunctionTok{fct\_recode}\NormalTok{(}
      \StringTok{"Hispanic"} \OtherTok{=} \StringTok{"1"}\NormalTok{,}
      \StringTok{"Non{-}Hispanic"} \OtherTok{=} \StringTok{"2"}\NormalTok{))}



\CommentTok{\#Re{-}coding marital status variable}
\NormalTok{data\_dec21 }\OtherTok{\textless{}{-}}\NormalTok{ data\_dec21 }\SpecialCharTok{\%\textgreater{}\%} 
  \FunctionTok{mutate}\NormalTok{(}\AttributeTok{Marital\_status =} \FunctionTok{factor}\NormalTok{(PEMARITL) }\SpecialCharTok{\%\textgreater{}\%} 
    \FunctionTok{fct\_recode}\NormalTok{(}
      \StringTok{"MARRIED {-} SPOUSE PRESENT"} \OtherTok{=} \StringTok{"1"}\NormalTok{,}
      \StringTok{"MARRIED {-} SPOUSE ABSENT"} \OtherTok{=} \StringTok{"2"}\NormalTok{,}
      \StringTok{"WIDOWED"} \OtherTok{=} \StringTok{"3"}\NormalTok{,}
      \StringTok{"DIVORCED"} \OtherTok{=} \StringTok{"4"}\NormalTok{,}
      \StringTok{"SEPARATED"} \OtherTok{=} \StringTok{"5"}\NormalTok{,}
      \StringTok{"NEVER MARRIED"} \OtherTok{=} \StringTok{"6"}\NormalTok{))}

\CommentTok{\#Re{-}coding highest complete degree variable}
\NormalTok{data\_dec21 }\OtherTok{\textless{}{-}}\NormalTok{ data\_dec21 }\SpecialCharTok{\%\textgreater{}\%} 
  \FunctionTok{mutate}\NormalTok{(}\AttributeTok{High\_Lvl\_Degree =} \FunctionTok{factor}\NormalTok{(PEEDUCA) }\SpecialCharTok{\%\textgreater{}\%} 
    \FunctionTok{fct\_recode}\NormalTok{(}
      \StringTok{"Less than H.S or GED"} \OtherTok{=} \StringTok{"31"}\NormalTok{,}
      \StringTok{"Less than H.S or GED"} \OtherTok{=} \StringTok{"32"}\NormalTok{,}
      \StringTok{"Less than H.S or GED"} \OtherTok{=} \StringTok{"33"}\NormalTok{,}
      \StringTok{"Less than H.S or GED"} \OtherTok{=} \StringTok{"34"}\NormalTok{,}
      \StringTok{"Less than H.S or GED"} \OtherTok{=} \StringTok{"35"}\NormalTok{,}
      \StringTok{"Less than H.S or GED"} \OtherTok{=} \StringTok{"36"}\NormalTok{,}
      \StringTok{"Less than H.S or GED"} \OtherTok{=} \StringTok{"37"}\NormalTok{,}
      \StringTok{"Less than H.S or GED"} \OtherTok{=} \StringTok{"31"}\NormalTok{,}
      \StringTok{"Less than H.S or GED"} \OtherTok{=} \StringTok{"38"}\NormalTok{,}
      \StringTok{"H.S or GED"} \OtherTok{=} \StringTok{"39"}\NormalTok{, }
      \StringTok{"H.S or GED"} \OtherTok{=} \StringTok{"40"}\NormalTok{,}
      \StringTok{"Associate\textquotesingle{}s"} \OtherTok{=} \StringTok{"41"}\NormalTok{,}
      \StringTok{"Associate\textquotesingle{}s"} \OtherTok{=} \StringTok{"42"}\NormalTok{,}
      \StringTok{"Bachelor\textquotesingle{}s"} \OtherTok{=} \StringTok{"43"}\NormalTok{,}
      \StringTok{"Master\textquotesingle{}s"} \OtherTok{=} \StringTok{"44"}\NormalTok{,}
      \StringTok{"Professional"} \OtherTok{=} \StringTok{"45"}\NormalTok{,}
      \StringTok{"Doctorate"} \OtherTok{=} \StringTok{"46"}\NormalTok{))}


\CommentTok{\#Makes state variable GESTFIPS into character}
\NormalTok{data\_dec21}\SpecialCharTok{$}\NormalTok{GESTFIPS }\OtherTok{\textless{}{-}} \FunctionTok{as.character}\NormalTok{(data\_dec21}\SpecialCharTok{$}\NormalTok{GESTFIPS)}
\CommentTok{\#summary(data\_dec21$GESTFIPS)}

\CommentTok{\#Changing data set to make a new column called "Stabr" which is the state abreviation}
\NormalTok{data\_dec21 }\OtherTok{\textless{}{-}}\NormalTok{ data\_dec21 }\SpecialCharTok{\%\textgreater{}\%}
                   \FunctionTok{mutate}\NormalTok{(}\AttributeTok{Stabr =} \FunctionTok{factor}\NormalTok{(GESTFIPS) }\SpecialCharTok{\%\textgreater{}\%}
                     \FunctionTok{fct\_recode}\NormalTok{( }
           \StringTok{"AL"} \OtherTok{=} \StringTok{"1"}\NormalTok{, }
           \StringTok{"AK"} \OtherTok{=} \StringTok{"2"}\NormalTok{, }
           \StringTok{"AZ"} \OtherTok{=} \StringTok{"4"}\NormalTok{,}
           \StringTok{"AR"} \OtherTok{=} \StringTok{"5"}\NormalTok{,}
           \StringTok{"CA"} \OtherTok{=} \StringTok{"6"}\NormalTok{,}
           \StringTok{"CO"} \OtherTok{=} \StringTok{"8"}\NormalTok{,}
           \StringTok{"CT"} \OtherTok{=} \StringTok{"9"}\NormalTok{,}
           \StringTok{"DE"} \OtherTok{=} \StringTok{"10"}\NormalTok{,}
           \StringTok{"DC"} \OtherTok{=} \StringTok{"11"}\NormalTok{,}
           \StringTok{"FL"} \OtherTok{=} \StringTok{"12"}\NormalTok{,}
           \StringTok{"GA"} \OtherTok{=} \StringTok{"13"}\NormalTok{,}
           \StringTok{"HI"} \OtherTok{=} \StringTok{"15"}\NormalTok{,}
           \StringTok{"ID"} \OtherTok{=} \StringTok{"16"}\NormalTok{,}
           \StringTok{"IL"} \OtherTok{=} \StringTok{"17"}\NormalTok{,}
           \StringTok{"IN"} \OtherTok{=} \StringTok{"18"}\NormalTok{,}
           \StringTok{"IA"} \OtherTok{=} \StringTok{"19"}\NormalTok{,}
           \StringTok{"KS"} \OtherTok{=} \StringTok{"20"}\NormalTok{,}
           \StringTok{"KY"} \OtherTok{=} \StringTok{"21"}\NormalTok{,}
           \StringTok{"LA"} \OtherTok{=} \StringTok{"22"}\NormalTok{,}
           \StringTok{"ME"} \OtherTok{=} \StringTok{"23"}\NormalTok{,}
           \StringTok{"MD"} \OtherTok{=} \StringTok{"24"}\NormalTok{,}
           \StringTok{"MA"} \OtherTok{=} \StringTok{"25"}\NormalTok{,}
           \StringTok{"MI"} \OtherTok{=} \StringTok{"26"}\NormalTok{,}
           \StringTok{"MN"} \OtherTok{=} \StringTok{"27"}\NormalTok{,}
           \StringTok{"MS"} \OtherTok{=} \StringTok{"28"}\NormalTok{,}
           \StringTok{"MO"} \OtherTok{=} \StringTok{"29"}\NormalTok{,}
           \StringTok{"MT"} \OtherTok{=} \StringTok{"30"}\NormalTok{,}
           \StringTok{"ME"} \OtherTok{=} \StringTok{"31"}\NormalTok{,}
           \StringTok{"MV"} \OtherTok{=} \StringTok{"32"}\NormalTok{,}
           \StringTok{"NH"} \OtherTok{=} \StringTok{"33"}\NormalTok{,}
           \StringTok{"NJ"} \OtherTok{=} \StringTok{"34"}\NormalTok{,}
           \StringTok{"NM"} \OtherTok{=} \StringTok{"35"}\NormalTok{,}
           \StringTok{"NY"} \OtherTok{=} \StringTok{"36"}\NormalTok{,}
           \StringTok{"NC"} \OtherTok{=} \StringTok{"37"}\NormalTok{,}
           \StringTok{"ND"} \OtherTok{=} \StringTok{"38"}\NormalTok{,}
           \StringTok{"OH"} \OtherTok{=} \StringTok{"39"}\NormalTok{,}
           \StringTok{"OK"} \OtherTok{=} \StringTok{"40"}\NormalTok{,}
           \StringTok{"OR"} \OtherTok{=} \StringTok{"41"}\NormalTok{,}
           \StringTok{"PA"} \OtherTok{=} \StringTok{"42"}\NormalTok{,}
           \StringTok{"RI"} \OtherTok{=} \StringTok{"44"}\NormalTok{,}
           \StringTok{"SC"} \OtherTok{=} \StringTok{"45"}\NormalTok{,}
           \StringTok{"SD"} \OtherTok{=} \StringTok{"46"}\NormalTok{,}
           \StringTok{"TN"} \OtherTok{=} \StringTok{"47"}\NormalTok{,}
           \StringTok{"TX"} \OtherTok{=} \StringTok{"48"}\NormalTok{,}
           \StringTok{"UT"} \OtherTok{=} \StringTok{"49"}\NormalTok{,}
           \StringTok{"VT"} \OtherTok{=} \StringTok{"50"}\NormalTok{,}
           \StringTok{"VA"} \OtherTok{=} \StringTok{"51"}\NormalTok{,}
           \StringTok{"WA"} \OtherTok{=} \StringTok{"53"}\NormalTok{,}
           \StringTok{"WV"} \OtherTok{=} \StringTok{"54"}\NormalTok{,}
           \StringTok{"WI"} \OtherTok{=} \StringTok{"55"}\NormalTok{,}
           \StringTok{"WY"} \OtherTok{=} \StringTok{"56"}\NormalTok{))}

\CommentTok{\#data\_dec21 \textless{}{-} data\_dec21 \%\textgreater{}\%}
\CommentTok{\#                   mutate(new\_name = factor(oldname) \%\textgreater{}\%}
\CommentTok{\#                     fct\_recode( }
\CommentTok{\#           "new\_level" = "old\_level")}



\CommentTok{\#Recoding Race Variable }

\NormalTok{data\_dec21}\SpecialCharTok{$}\NormalTok{PTDTRACE }\OtherTok{\textless{}{-}} \FunctionTok{as.character}\NormalTok{(data\_dec21}\SpecialCharTok{$}\NormalTok{PTDTRACE)}

\NormalTok{data\_dec21 }\OtherTok{\textless{}{-}}\NormalTok{ data\_dec21 }\SpecialCharTok{\%\textgreater{}\%} 
  \FunctionTok{mutate}\NormalTok{(}\AttributeTok{Race =} \FunctionTok{factor}\NormalTok{(PTDTRACE) }\SpecialCharTok{\%\textgreater{}\%} 
    \FunctionTok{fct\_recode}\NormalTok{(}
      \StringTok{"White Only"} \OtherTok{=} \StringTok{"1"}\NormalTok{,}
      \StringTok{"Black Only"} \OtherTok{=} \StringTok{"2"}\NormalTok{,}
      \StringTok{"American Indian, Alaskan Native Only"} \OtherTok{=} \StringTok{"3"}\NormalTok{,}
      \StringTok{"Aisan Only"} \OtherTok{=} \StringTok{"4"}\NormalTok{,}
      \StringTok{"Hawaiian/Pacific Islander Only"} \OtherTok{=} \StringTok{"5"}\NormalTok{,}
      \StringTok{"White{-}Black"} \OtherTok{=} \StringTok{"6"}\NormalTok{,}
      \StringTok{"White{-}AI"} \OtherTok{=} \StringTok{"7"}\NormalTok{,}
      \StringTok{"White{-}Asian"} \OtherTok{=} \StringTok{"8"}\NormalTok{, }
      \StringTok{"White{-}HP"} \OtherTok{=} \StringTok{"9"}\NormalTok{,}
      \StringTok{"Black{-}AI"} \OtherTok{=} \StringTok{"10"}\NormalTok{,}
      \StringTok{"Black{-}Asian"} \OtherTok{=} \StringTok{"11"}\NormalTok{,}
      \StringTok{"Black{-}HP"} \OtherTok{=} \StringTok{"12"}\NormalTok{,}
      \StringTok{"AI{-}Asian"} \OtherTok{=} \StringTok{"13"}\NormalTok{,}
      \StringTok{"AI{-}HP"} \OtherTok{=} \StringTok{"14"}\NormalTok{,}
      \StringTok{"Asian{-}HP"} \OtherTok{=} \StringTok{"15"}\NormalTok{,}
      \StringTok{"3+ mixed{-}race"} \OtherTok{=} \StringTok{"16"}\NormalTok{,}
      \StringTok{"3+ mixed{-}race"} \OtherTok{=} \StringTok{"17"}\NormalTok{, }
      \StringTok{"3+ mixed{-}race"} \OtherTok{=} \StringTok{"18"}\NormalTok{, }
      \StringTok{"3+ mixed{-}race"} \OtherTok{=} \StringTok{"19"}\NormalTok{, }
      \StringTok{"3+ mixed{-}race"} \OtherTok{=} \StringTok{"20"}\NormalTok{, }
      \StringTok{"3+ mixed{-}race"} \OtherTok{=} \StringTok{"21"}\NormalTok{, }
      \StringTok{"3+ mixed{-}race"} \OtherTok{=} \StringTok{"22"}\NormalTok{, }
      \StringTok{"3+ mixed{-}race"} \OtherTok{=} \StringTok{"23"}\NormalTok{, }
      \StringTok{"3+ mixed{-}race"} \OtherTok{=} \StringTok{"24"}\NormalTok{, }
      \StringTok{"3+ mixed{-}race"} \OtherTok{=} \StringTok{"25"}\NormalTok{, }
      \StringTok{"3+ mixed{-}race"} \OtherTok{=} \StringTok{"26"}\NormalTok{))}
\end{Highlighting}
\end{Shaded}

\begin{verbatim}
## Warning: Unknown levels in `f`: 22, 24
\end{verbatim}

\begin{Shaded}
\begin{Highlighting}[]
\CommentTok{\#Mutating to re{-}code Sex Variable }
\NormalTok{data\_dec21 }\OtherTok{\textless{}{-}}\NormalTok{ data\_dec21 }\SpecialCharTok{\%\textgreater{}\%} 
  \FunctionTok{mutate}\NormalTok{(}\AttributeTok{Sex =} \FunctionTok{factor}\NormalTok{(PESEX) }\SpecialCharTok{\%\textgreater{}\%} 
    \FunctionTok{fct\_recode}\NormalTok{(}
      \StringTok{"Male"} \OtherTok{=} \StringTok{"1"}\NormalTok{,}
      \StringTok{"Female"} \OtherTok{=} \StringTok{"2"}\NormalTok{))     }

\CommentTok{\#Mutating to recode Age variable}
\NormalTok{ data\_dec21 }\OtherTok{\textless{}{-}}\NormalTok{ data\_dec21 }\SpecialCharTok{\%\textgreater{}\%} 
  \FunctionTok{mutate}\NormalTok{(}\AttributeTok{Age =} \FunctionTok{as.numeric}\NormalTok{(}\FunctionTok{factor}\NormalTok{(PRTAGE)))}
\end{Highlighting}
\end{Shaded}

Re-coding Food Security Measures

\begin{Shaded}
\begin{Highlighting}[]
\CommentTok{\#Mutating to re code Main food insecurity measure variable}
\NormalTok{ data\_dec21 }\OtherTok{\textless{}{-}}\NormalTok{ data\_dec21 }\SpecialCharTok{\%\textgreater{}\%} 
  \FunctionTok{mutate}\NormalTok{(}\AttributeTok{twelveM\_Household\_FI =} \FunctionTok{factor}\NormalTok{(HRFS12M1) }\SpecialCharTok{\%\textgreater{}\%} 
           \FunctionTok{fct\_recode}\NormalTok{(}
             \StringTok{"Food Secure High or Marginal Food Security"} \OtherTok{=} \StringTok{"1"}\NormalTok{,}
             \StringTok{"Low Food Security"} \OtherTok{=} \StringTok{"2"}\NormalTok{,}
             \StringTok{"Very Low Food Security"} \OtherTok{=} \StringTok{"3"}\NormalTok{))}
 
 \CommentTok{\#Mutating to recode food insecurity scale rasch variable}
\NormalTok{ data\_dec21 }\OtherTok{\textless{}{-}}\NormalTok{ data\_dec21 }\SpecialCharTok{\%\textgreater{}\%} 
  \FunctionTok{mutate}\NormalTok{(}\AttributeTok{twelveM\_Household\_FI\_rasch =} \FunctionTok{factor}\NormalTok{(HRFS12M4))}
 \CommentTok{\#Making the rasch variable numeric so it can be graphed}
 
\NormalTok{ data\_dec21}\SpecialCharTok{$}\NormalTok{twelveM\_Household\_FI\_rasch }\OtherTok{\textless{}{-}} \FunctionTok{as.numeric}\NormalTok{(}\FunctionTok{as.character}\NormalTok{(data\_dec21}\SpecialCharTok{$}\NormalTok{twelveM\_Household\_FI\_rasch))}
 
 
 \CommentTok{\#Mutating to re code food insecurity variable which combines the main food insecurity measure into food insecure or not}
\NormalTok{ data\_dec21 }\OtherTok{\textless{}{-}}\NormalTok{ data\_dec21 }\SpecialCharTok{\%\textgreater{}\%} 
  \FunctionTok{mutate}\NormalTok{(}\AttributeTok{FI\_orNot =} \FunctionTok{factor}\NormalTok{(HRFS12MD) }\SpecialCharTok{\%\textgreater{}\%} 
           \FunctionTok{fct\_recode}\NormalTok{(}
             \StringTok{"High Food Security"} \OtherTok{=} \StringTok{"1"}\NormalTok{,}
             \StringTok{"Food Insecure"} \OtherTok{=} \StringTok{"2"}\NormalTok{,}
             \StringTok{"Food Insecure"} \OtherTok{=} \StringTok{"3"}\NormalTok{,}
             \StringTok{"Food Insecure"} \OtherTok{=} \StringTok{"4"}\NormalTok{))}
\end{Highlighting}
\end{Shaded}

Re-selecting Re-coded Variables for new data set

\begin{Shaded}
\begin{Highlighting}[]
\NormalTok{data\_dec21 }\OtherTok{\textless{}{-}} \FunctionTok{select}\NormalTok{(data\_dec21,FIPS\_code, Stabr, Age, Race, Sex, Hispanic, }
\NormalTok{                     Marital\_status, High\_Lvl\_Degree, twelveM\_Household\_FI, twelveM\_Household\_FI\_rasch, FI\_orNot)}
\end{Highlighting}
\end{Shaded}

Summary Statistics of Food Security Measures

\begin{Shaded}
\begin{Highlighting}[]
\CommentTok{\#Food Insecurity Main measure}
\NormalTok{data\_dec21 }\SpecialCharTok{\%\textgreater{}\%}  \FunctionTok{filter}\NormalTok{(}\SpecialCharTok{!}\FunctionTok{is.na}\NormalTok{(twelveM\_Household\_FI)) }\SpecialCharTok{\%\textgreater{}\%} \FunctionTok{ggplot}\NormalTok{(}\FunctionTok{aes}\NormalTok{(}\AttributeTok{x =}\NormalTok{ twelveM\_Household\_FI)) }\SpecialCharTok{+}
  \FunctionTok{geom\_bar}\NormalTok{() }\SpecialCharTok{+}
  \FunctionTok{theme}\NormalTok{(}\AttributeTok{axis.text.x =} \FunctionTok{element\_text}\NormalTok{(}\AttributeTok{angle=}\DecValTok{45}\NormalTok{, }\AttributeTok{vjust=}\DecValTok{1}\NormalTok{, }\AttributeTok{hjust=}\DecValTok{1}\NormalTok{)) }\SpecialCharTok{+} \CommentTok{\#45 degree angle labels}
  \FunctionTok{labs}\NormalTok{(}\AttributeTok{title =} \StringTok{"Main Food Insecurity Measure Counts"}\NormalTok{, }\AttributeTok{x =} \StringTok{"Food Insecurity Level"}\NormalTok{, }\AttributeTok{caption =} \StringTok{"Source: Current Population Survey Food Security Supplements"}\NormalTok{)}
\end{Highlighting}
\end{Shaded}

\includegraphics{Final_Activity_v3_files/figure-latex/unnamed-chunk-5-1.pdf}

\begin{Shaded}
\begin{Highlighting}[]
\CommentTok{\#Food Insecurity}
\NormalTok{data\_dec21 }\SpecialCharTok{\%\textgreater{}\%}  \FunctionTok{filter}\NormalTok{(}\SpecialCharTok{!}\FunctionTok{is.na}\NormalTok{(FI\_orNot)) }\SpecialCharTok{\%\textgreater{}\%} \FunctionTok{ggplot}\NormalTok{(}\FunctionTok{aes}\NormalTok{(}\AttributeTok{x =}\NormalTok{ FI\_orNot)) }\SpecialCharTok{+}
  \FunctionTok{geom\_bar}\NormalTok{() }\SpecialCharTok{+}
  \FunctionTok{theme}\NormalTok{(}\AttributeTok{axis.text.x =} \FunctionTok{element\_text}\NormalTok{(}\AttributeTok{angle=}\DecValTok{45}\NormalTok{, }\AttributeTok{vjust=}\DecValTok{1}\NormalTok{, }\AttributeTok{hjust=}\DecValTok{1}\NormalTok{)) }\SpecialCharTok{+} \CommentTok{\#45 degree angle labels}
  \FunctionTok{labs}\NormalTok{(}\AttributeTok{title =} \StringTok{"High Food Insecurity or Not Counts"}\NormalTok{, }\AttributeTok{x =} \StringTok{"Food Insecurity Level"}\NormalTok{, }\AttributeTok{caption =} \StringTok{"Source: Current Population Survey Food Security Supplements"}\NormalTok{)}
\end{Highlighting}
\end{Shaded}

\includegraphics{Final_Activity_v3_files/figure-latex/unnamed-chunk-5-2.pdf}

In both categorizations of food insecurity, most people are food secure.

Based on the Counts of Race I have decided to make some of the values
mixed race rather than their own category due to their relatively
extremely low numbers:

Other Mixed Race: Black-Asian AI-HP Black-HP Asian-HP White-HP Black-AI
AI-Asian 3+ mixed race

White-Black

White Asian

White-AI

White ONly

Asian Only

Black Only

American Indian, Alaskan Native Only

Hawaiian/Pacific Islander Only

Re-coding the Race variable:

\begin{Shaded}
\begin{Highlighting}[]
\NormalTok{data\_dec21 }\OtherTok{\textless{}{-}}\NormalTok{ data\_dec21 }\SpecialCharTok{\%\textgreater{}\%} 
  \FunctionTok{mutate}\NormalTok{(}\AttributeTok{Race =} \FunctionTok{factor}\NormalTok{(Race) }\SpecialCharTok{\%\textgreater{}\%} 
    \FunctionTok{fct\_recode}\NormalTok{(}
      \StringTok{"Other Mixed Race"} \OtherTok{=} \StringTok{"Black{-}Asian"}\NormalTok{,}
      \StringTok{"Other Mixed Race"} \OtherTok{=} \StringTok{"AI{-}HP"}\NormalTok{,}
      \StringTok{"Other Mixed Race"} \OtherTok{=} \StringTok{"Black{-}HP"}\NormalTok{,}
      \StringTok{"Other Mixed Race"} \OtherTok{=} \StringTok{"Asian{-}HP"}\NormalTok{,}
      \StringTok{"Other Mixed Race"} \OtherTok{=} \StringTok{"White{-}HP"}\NormalTok{,}
      \StringTok{"Other Mixed Race"} \OtherTok{=} \StringTok{"White{-}HP"}\NormalTok{,}
      \StringTok{"Other Mixed Race"} \OtherTok{=} \StringTok{"Black{-}AI"}\NormalTok{,}
      \StringTok{"Other Mixed Race"} \OtherTok{=} \StringTok{"AI{-}Asian"}\NormalTok{,}
      \StringTok{"Other Mixed Race"} \OtherTok{=} \StringTok{"3+ mixed{-}race"}\NormalTok{))}


\NormalTok{data\_dec21 }\SpecialCharTok{\%\textgreater{}\%} \FunctionTok{count}\NormalTok{(Race)}
\end{Highlighting}
\end{Shaded}

\begin{verbatim}
##                                    Race     n
## 1                            White Only 80955
## 2                      Other Mixed Race   557
## 3                            Black Only 11100
## 4  American Indian, Alaskan Native Only  1320
## 5                            Aisan Only  5768
## 6        Hawaiian/Pacific Islander Only   483
## 7                           White-Black   753
## 8                              White-AI   616
## 9                           White-Asian   473
## 10                                 <NA> 25464
\end{verbatim}

\#\#Data Dictionary

\begin{Shaded}
\begin{Highlighting}[]
\FunctionTok{skim}\NormalTok{(data\_dec21)}
\end{Highlighting}
\end{Shaded}

\begin{longtable}[]{@{}ll@{}}
\caption{Data summary}\tabularnewline
\toprule()
\endhead
Name & data\_dec21 \\
Number of rows & 127489 \\
Number of columns & 11 \\
\_\_\_\_\_\_\_\_\_\_\_\_\_\_\_\_\_\_\_\_\_\_\_ & \\
Column type frequency: & \\
character & 1 \\
factor & 8 \\
numeric & 2 \\
\_\_\_\_\_\_\_\_\_\_\_\_\_\_\_\_\_\_\_\_\_\_\_\_ & \\
Group variables & None \\
\bottomrule()
\end{longtable}

\textbf{Variable type: character}

\begin{longtable}[]{@{}
  >{\raggedright\arraybackslash}p{(\columnwidth - 14\tabcolsep) * \real{0.1944}}
  >{\raggedleft\arraybackslash}p{(\columnwidth - 14\tabcolsep) * \real{0.1389}}
  >{\raggedleft\arraybackslash}p{(\columnwidth - 14\tabcolsep) * \real{0.1944}}
  >{\raggedleft\arraybackslash}p{(\columnwidth - 14\tabcolsep) * \real{0.0556}}
  >{\raggedleft\arraybackslash}p{(\columnwidth - 14\tabcolsep) * \real{0.0556}}
  >{\raggedleft\arraybackslash}p{(\columnwidth - 14\tabcolsep) * \real{0.0833}}
  >{\raggedleft\arraybackslash}p{(\columnwidth - 14\tabcolsep) * \real{0.1250}}
  >{\raggedleft\arraybackslash}p{(\columnwidth - 14\tabcolsep) * \real{0.1528}}@{}}
\toprule()
\begin{minipage}[b]{\linewidth}\raggedright
skim\_variable
\end{minipage} & \begin{minipage}[b]{\linewidth}\raggedleft
n\_missing
\end{minipage} & \begin{minipage}[b]{\linewidth}\raggedleft
complete\_rate
\end{minipage} & \begin{minipage}[b]{\linewidth}\raggedleft
min
\end{minipage} & \begin{minipage}[b]{\linewidth}\raggedleft
max
\end{minipage} & \begin{minipage}[b]{\linewidth}\raggedleft
empty
\end{minipage} & \begin{minipage}[b]{\linewidth}\raggedleft
n\_unique
\end{minipage} & \begin{minipage}[b]{\linewidth}\raggedleft
whitespace
\end{minipage} \\
\midrule()
\endhead
FIPS\_code & 0 & 1 & 5 & 5 & 0 & 329 & 0 \\
\bottomrule()
\end{longtable}

\textbf{Variable type: factor}

\begin{longtable}[]{@{}
  >{\raggedright\arraybackslash}p{(\columnwidth - 10\tabcolsep) * \real{0.1944}}
  >{\raggedleft\arraybackslash}p{(\columnwidth - 10\tabcolsep) * \real{0.0926}}
  >{\raggedleft\arraybackslash}p{(\columnwidth - 10\tabcolsep) * \real{0.1296}}
  >{\raggedright\arraybackslash}p{(\columnwidth - 10\tabcolsep) * \real{0.0741}}
  >{\raggedleft\arraybackslash}p{(\columnwidth - 10\tabcolsep) * \real{0.0833}}
  >{\raggedright\arraybackslash}p{(\columnwidth - 10\tabcolsep) * \real{0.4259}}@{}}
\toprule()
\begin{minipage}[b]{\linewidth}\raggedright
skim\_variable
\end{minipage} & \begin{minipage}[b]{\linewidth}\raggedleft
n\_missing
\end{minipage} & \begin{minipage}[b]{\linewidth}\raggedleft
complete\_rate
\end{minipage} & \begin{minipage}[b]{\linewidth}\raggedright
ordered
\end{minipage} & \begin{minipage}[b]{\linewidth}\raggedleft
n\_unique
\end{minipage} & \begin{minipage}[b]{\linewidth}\raggedright
top\_counts
\end{minipage} \\
\midrule()
\endhead
Stabr & 0 & 1.00 & FALSE & 50 & CA: 10189, TX: 7721, FL: 5833, NY:
4864 \\
Race & 25464 & 0.80 & FALSE & 9 & Whi: 80955, Bla: 11100, Ais: 5768,
Ame: 1320 \\
Sex & 25464 & 0.80 & FALSE & 2 & Fem: 52162, Mal: 49863 \\
Hispanic & 25464 & 0.80 & FALSE & 2 & Non: 86707, His: 15318 \\
Marital\_status & 43334 & 0.66 & FALSE & 6 & MAR: 41972, NEV: 25594,
DIV: 8623, WID: 5490 \\
High\_Lvl\_Degree & 43334 & 0.66 & FALSE & 7 & H.S: 37447, Bac: 17656,
Les: 10545, Ass: 8093 \\
twelveM\_Household\_FI & 56017 & 0.56 & FALSE & 3 & Foo: 64343, Low:
4820, Ver: 2309 \\
FI\_orNot & 56017 & 0.56 & FALSE & 2 & Hig: 58901, Foo: 12571 \\
\bottomrule()
\end{longtable}

\textbf{Variable type: numeric}

\begin{longtable}[]{@{}
  >{\raggedright\arraybackslash}p{(\columnwidth - 20\tabcolsep) * \real{0.2935}}
  >{\raggedleft\arraybackslash}p{(\columnwidth - 20\tabcolsep) * \real{0.1087}}
  >{\raggedleft\arraybackslash}p{(\columnwidth - 20\tabcolsep) * \real{0.1522}}
  >{\raggedleft\arraybackslash}p{(\columnwidth - 20\tabcolsep) * \real{0.0761}}
  >{\raggedleft\arraybackslash}p{(\columnwidth - 20\tabcolsep) * \real{0.0761}}
  >{\raggedleft\arraybackslash}p{(\columnwidth - 20\tabcolsep) * \real{0.0435}}
  >{\raggedleft\arraybackslash}p{(\columnwidth - 20\tabcolsep) * \real{0.0435}}
  >{\raggedleft\arraybackslash}p{(\columnwidth - 20\tabcolsep) * \real{0.0435}}
  >{\raggedleft\arraybackslash}p{(\columnwidth - 20\tabcolsep) * \real{0.0435}}
  >{\raggedleft\arraybackslash}p{(\columnwidth - 20\tabcolsep) * \real{0.0543}}
  >{\raggedright\arraybackslash}p{(\columnwidth - 20\tabcolsep) * \real{0.0652}}@{}}
\toprule()
\begin{minipage}[b]{\linewidth}\raggedright
skim\_variable
\end{minipage} & \begin{minipage}[b]{\linewidth}\raggedleft
n\_missing
\end{minipage} & \begin{minipage}[b]{\linewidth}\raggedleft
complete\_rate
\end{minipage} & \begin{minipage}[b]{\linewidth}\raggedleft
mean
\end{minipage} & \begin{minipage}[b]{\linewidth}\raggedleft
sd
\end{minipage} & \begin{minipage}[b]{\linewidth}\raggedleft
p0
\end{minipage} & \begin{minipage}[b]{\linewidth}\raggedleft
p25
\end{minipage} & \begin{minipage}[b]{\linewidth}\raggedleft
p50
\end{minipage} & \begin{minipage}[b]{\linewidth}\raggedleft
p75
\end{minipage} & \begin{minipage}[b]{\linewidth}\raggedleft
p100
\end{minipage} & \begin{minipage}[b]{\linewidth}\raggedright
hist
\end{minipage} \\
\midrule()
\endhead
Age & 25464 & 0.8 & 41.98 & 23.46 & 1 & 22 & 42 & 62 & 82 & ▇▇▇▇▇ \\
twelveM\_Household\_FI\_rasch & 114918 & 0.1 & 416.91 & 241.60 & 143 &
172 & 340 & 543 & 1303 & ▇▅▂▁▁ \\
\bottomrule()
\end{longtable}

\begin{Shaded}
\begin{Highlighting}[]
\FunctionTok{glimpse}\NormalTok{(data\_dec21)}
\end{Highlighting}
\end{Shaded}

\begin{verbatim}
## Rows: 127,489
## Columns: 11
## $ FIPS_code                  <chr> "01003", "01003", "01003", "01003", "01003"~
## $ Stabr                      <fct> AL, AL, AL, AL, AL, AL, AL, AL, AL, AL, AL,~
## $ Age                        <dbl> 56, 25, 22, 16, 38, 13, 23, 24, 47, 47, 69,~
## $ Race                       <fct> "White Only", "White Only", "White Only", "~
## $ Sex                        <fct> Male, Female, Male, Female, Male, Female, M~
## $ Hispanic                   <fct> Non-Hispanic, Non-Hispanic, Non-Hispanic, N~
## $ Marital_status             <fct> NEVER MARRIED, MARRIED - SPOUSE PRESENT, MA~
## $ High_Lvl_Degree            <fct> Bachelor's, Bachelor's, Bachelor's, H.S or ~
## $ twelveM_Household_FI       <fct> Food Secure High or Marginal Food Security,~
## $ twelveM_Household_FI_rasch <dbl> NA, NA, NA, NA, NA, NA, NA, NA, NA, NA, NA,~
## $ FI_orNot                   <fct> High Food Security, NA, NA, NA, NA, NA, Hig~
\end{verbatim}

\begin{Shaded}
\begin{Highlighting}[]
\NormalTok{dataDictionary }\OtherTok{\textless{}{-}} \FunctionTok{tibble}\NormalTok{(}\AttributeTok{Variable =} \FunctionTok{colnames}\NormalTok{(data\_dec21),}
                         \AttributeTok{Description =} \FunctionTok{c}\NormalTok{(}\StringTok{"Unique County Code"}\NormalTok{, }\StringTok{"Abreviated State Name"}\NormalTok{, }\StringTok{"Age"}\NormalTok{, }\StringTok{"Hispanic"}\NormalTok{, }\StringTok{"Race"}\NormalTok{, }\StringTok{"Sex"}\NormalTok{, }\StringTok{"Marital Status"}\NormalTok{,}\StringTok{"Highest Level Ed. Degree Completed"}\NormalTok{, }\StringTok{"Household Food Security Scale, 12{-}Month Reference Period"}\NormalTok{, }\StringTok{"Food Security Rasch Scale Score, 12{-}Month Recall"}\NormalTok{, }\StringTok{"High Food Insecurity or not"}\NormalTok{),}
                         \AttributeTok{Type =} \FunctionTok{map\_chr}\NormalTok{(data\_dec21, }\AttributeTok{.f =} \ControlFlowTok{function}\NormalTok{(x)\{}\FunctionTok{typeof}\NormalTok{(x)[}\DecValTok{1}\NormalTok{]\}),}
                         \AttributeTok{Class =} \FunctionTok{map\_chr}\NormalTok{(data\_dec21, }\AttributeTok{.f =} \ControlFlowTok{function}\NormalTok{(x)\{}\FunctionTok{class}\NormalTok{(x)[}\DecValTok{1}\NormalTok{]\}))}

\NormalTok{knitr}\SpecialCharTok{::}\FunctionTok{kable}\NormalTok{(dataDictionary)}
\end{Highlighting}
\end{Shaded}

\begin{longtable}[]{@{}
  >{\raggedright\arraybackslash}p{(\columnwidth - 6\tabcolsep) * \real{0.2596}}
  >{\raggedright\arraybackslash}p{(\columnwidth - 6\tabcolsep) * \real{0.5481}}
  >{\raggedright\arraybackslash}p{(\columnwidth - 6\tabcolsep) * \real{0.0962}}
  >{\raggedright\arraybackslash}p{(\columnwidth - 6\tabcolsep) * \real{0.0962}}@{}}
\toprule()
\begin{minipage}[b]{\linewidth}\raggedright
Variable
\end{minipage} & \begin{minipage}[b]{\linewidth}\raggedright
Description
\end{minipage} & \begin{minipage}[b]{\linewidth}\raggedright
Type
\end{minipage} & \begin{minipage}[b]{\linewidth}\raggedright
Class
\end{minipage} \\
\midrule()
\endhead
FIPS\_code & Unique County Code & character & character \\
Stabr & Abreviated State Name & integer & factor \\
Age & Age & double & numeric \\
Race & Hispanic & integer & factor \\
Sex & Race & integer & factor \\
Hispanic & Sex & integer & factor \\
Marital\_status & Marital Status & integer & factor \\
High\_Lvl\_Degree & Highest Level Ed. Degree Completed & integer &
factor \\
twelveM\_Household\_FI & Household Food Security Scale, 12-Month
Reference Period & integer & factor \\
twelveM\_Household\_FI\_rasch & Food Security Rasch Scale Score,
12-Month Recall & double & numeric \\
FI\_orNot & High Food Insecurity or not & integer & factor \\
\bottomrule()
\end{longtable}

\hypertarget{summary-statistics---food-insecurity-or-not}{%
\subsection{Summary Statistics - Food Insecurity or
Not}\label{summary-statistics---food-insecurity-or-not}}

A breakdown by percentage of each group that is food insecure or not

\begin{Shaded}
\begin{Highlighting}[]
\NormalTok{data\_dec21 }\SpecialCharTok{\%\textgreater{}\%} \FunctionTok{filter}\NormalTok{(}\SpecialCharTok{!}\FunctionTok{is.na}\NormalTok{(FI\_orNot)) }\SpecialCharTok{\%\textgreater{}\%} 
\FunctionTok{tabyl}\NormalTok{(High\_Lvl\_Degree, FI\_orNot) }\SpecialCharTok{\%\textgreater{}\%} 
  \FunctionTok{adorn\_percentages}\NormalTok{(}\StringTok{"row"}\NormalTok{) }\SpecialCharTok{\%\textgreater{}\%} 
  \FunctionTok{adorn\_pct\_formatting}\NormalTok{(}\AttributeTok{digits =} \DecValTok{1}\NormalTok{)}
\end{Highlighting}
\end{Shaded}

\begin{verbatim}
##       High_Lvl_Degree High Food Security Food Insecure
##  Less than H.S or GED              70.0%         30.0%
##            H.S or GED              79.2%         20.8%
##           Associate's              84.2%         15.8%
##            Bachelor's              92.1%          7.9%
##              Master's              94.7%          5.3%
##          Professional              96.5%          3.5%
##             Doctorate              97.1%          2.9%
##                  <NA>              77.3%         22.7%
\end{verbatim}

\begin{Shaded}
\begin{Highlighting}[]
\NormalTok{data\_dec21 }\SpecialCharTok{\%\textgreater{}\%} \FunctionTok{filter}\NormalTok{(}\SpecialCharTok{!}\FunctionTok{is.na}\NormalTok{(FI\_orNot)) }\SpecialCharTok{\%\textgreater{}\%} 
\FunctionTok{tabyl}\NormalTok{(Race, FI\_orNot) }\SpecialCharTok{\%\textgreater{}\%} 
  \FunctionTok{adorn\_percentages}\NormalTok{(}\StringTok{"row"}\NormalTok{) }\SpecialCharTok{\%\textgreater{}\%} 
  \FunctionTok{adorn\_pct\_formatting}\NormalTok{(}\AttributeTok{digits =} \DecValTok{1}\NormalTok{)}
\end{Highlighting}
\end{Shaded}

\begin{verbatim}
##                                  Race High Food Security Food Insecure
##                            White Only              84.7%         15.3%
##                      Other Mixed Race              69.6%         30.4%
##                            Black Only              67.9%         32.1%
##  American Indian, Alaskan Native Only              64.5%         35.5%
##                            Aisan Only              85.3%         14.7%
##        Hawaiian/Pacific Islander Only              69.9%         30.1%
##                           White-Black              69.1%         30.9%
##                              White-AI              68.4%         31.6%
##                           White-Asian              86.3%         13.7%
\end{verbatim}

\begin{Shaded}
\begin{Highlighting}[]
\NormalTok{data\_dec21 }\SpecialCharTok{\%\textgreater{}\%} \FunctionTok{filter}\NormalTok{(}\SpecialCharTok{!}\FunctionTok{is.na}\NormalTok{(FI\_orNot)) }\SpecialCharTok{\%\textgreater{}\%} 
\FunctionTok{tabyl}\NormalTok{(Hispanic, FI\_orNot) }\SpecialCharTok{\%\textgreater{}\%} 
  \FunctionTok{adorn\_percentages}\NormalTok{(}\StringTok{"row"}\NormalTok{) }\SpecialCharTok{\%\textgreater{}\%} 
  \FunctionTok{adorn\_pct\_formatting}\NormalTok{(}\AttributeTok{digits =} \DecValTok{1}\NormalTok{)}
\end{Highlighting}
\end{Shaded}

\begin{verbatim}
##      Hispanic High Food Security Food Insecure
##      Hispanic              70.9%         29.1%
##  Non-Hispanic              84.4%         15.6%
\end{verbatim}

\begin{Shaded}
\begin{Highlighting}[]
\NormalTok{data\_dec21 }\SpecialCharTok{\%\textgreater{}\%} \FunctionTok{filter}\NormalTok{(}\SpecialCharTok{!}\FunctionTok{is.na}\NormalTok{(FI\_orNot)) }\SpecialCharTok{\%\textgreater{}\%} 
\FunctionTok{tabyl}\NormalTok{(Stabr, FI\_orNot) }\SpecialCharTok{\%\textgreater{}\%} 
  \FunctionTok{adorn\_percentages}\NormalTok{(}\StringTok{"row"}\NormalTok{) }\SpecialCharTok{\%\textgreater{}\%} 
  \FunctionTok{adorn\_pct\_formatting}\NormalTok{(}\AttributeTok{digits =} \DecValTok{1}\NormalTok{)}
\end{Highlighting}
\end{Shaded}

\begin{verbatim}
##  Stabr High Food Security Food Insecure
##     AL              83.8%         16.2%
##     DE              81.4%         18.6%
##     DC              85.1%         14.9%
##     FL              82.6%         17.4%
##     GA              84.2%         15.8%
##     HI              83.7%         16.3%
##     ID              81.5%         18.5%
##     IL              84.4%         15.6%
##     IN              85.8%         14.2%
##     IA              82.0%         18.0%
##     AK              81.0%         19.0%
##     KS              83.9%         16.1%
##     KY              83.5%         16.5%
##     LA              77.3%         22.7%
##     ME              83.3%         16.7%
##     MD              85.5%         14.5%
##     MA              85.6%         14.4%
##     MI              81.7%         18.3%
##     MN              88.3%         11.7%
##     MS              74.6%         25.4%
##     MO              82.6%         17.4%
##     MT              82.9%         17.1%
##     MV              82.1%         17.9%
##     NH              91.0%          9.0%
##     NJ              83.4%         16.6%
##     NM              79.7%         20.3%
##     NY              82.6%         17.4%
##     NC              84.6%         15.4%
##     ND              88.9%         11.1%
##     OH              82.2%         17.8%
##     AZ              85.2%         14.8%
##     OK              72.3%         27.7%
##     OR              82.4%         17.6%
##     PA              85.5%         14.5%
##     RI              87.6%         12.4%
##     SC              79.8%         20.2%
##     SD              82.5%         17.5%
##     TN              84.8%         15.2%
##     TX              75.7%         24.3%
##     UT              82.3%         17.7%
##     AR              76.8%         23.2%
##     VT              87.7%         12.3%
##     VA              85.3%         14.7%
##     WA              86.5%         13.5%
##     WV              77.8%         22.2%
##     WI              85.1%         14.9%
##     WY              83.5%         16.5%
##     CA              80.8%         19.2%
##     CO              82.7%         17.3%
##     CT              83.0%         17.0%
\end{verbatim}

\begin{Shaded}
\begin{Highlighting}[]
\NormalTok{data\_dec21 }\SpecialCharTok{\%\textgreater{}\%} \FunctionTok{filter}\NormalTok{(}\SpecialCharTok{!}\FunctionTok{is.na}\NormalTok{(FI\_orNot)) }\SpecialCharTok{\%\textgreater{}\%} 
\FunctionTok{tabyl}\NormalTok{(Marital\_status, FI\_orNot) }\SpecialCharTok{\%\textgreater{}\%} 
  \FunctionTok{adorn\_percentages}\NormalTok{(}\StringTok{"row"}\NormalTok{) }\SpecialCharTok{\%\textgreater{}\%} 
  \FunctionTok{adorn\_pct\_formatting}\NormalTok{(}\AttributeTok{digits =} \DecValTok{1}\NormalTok{)}
\end{Highlighting}
\end{Shaded}

\begin{verbatim}
##            Marital_status High Food Security Food Insecure
##  MARRIED - SPOUSE PRESENT              89.0%         11.0%
##   MARRIED - SPOUSE ABSENT              76.9%         23.1%
##                   WIDOWED              83.3%         16.7%
##                  DIVORCED              76.8%         23.2%
##                 SEPARATED              66.9%         33.1%
##             NEVER MARRIED              77.8%         22.2%
##                      <NA>              77.3%         22.7%
\end{verbatim}

\begin{Shaded}
\begin{Highlighting}[]
\NormalTok{data\_dec21 }\SpecialCharTok{\%\textgreater{}\%} \FunctionTok{filter}\NormalTok{(}\SpecialCharTok{!}\FunctionTok{is.na}\NormalTok{(FI\_orNot)) }\SpecialCharTok{\%\textgreater{}\%} 
\FunctionTok{tabyl}\NormalTok{(High\_Lvl\_Degree, FI\_orNot) }\SpecialCharTok{\%\textgreater{}\%} 
  \FunctionTok{adorn\_percentages}\NormalTok{(}\StringTok{"row"}\NormalTok{) }\SpecialCharTok{\%\textgreater{}\%} 
  \FunctionTok{adorn\_pct\_formatting}\NormalTok{(}\AttributeTok{digits =} \DecValTok{1}\NormalTok{)}
\end{Highlighting}
\end{Shaded}

\begin{verbatim}
##       High_Lvl_Degree High Food Security Food Insecure
##  Less than H.S or GED              70.0%         30.0%
##            H.S or GED              79.2%         20.8%
##           Associate's              84.2%         15.8%
##            Bachelor's              92.1%          7.9%
##              Master's              94.7%          5.3%
##          Professional              96.5%          3.5%
##             Doctorate              97.1%          2.9%
##                  <NA>              77.3%         22.7%
\end{verbatim}

\begin{Shaded}
\begin{Highlighting}[]
\NormalTok{data\_dec21 }\SpecialCharTok{\%\textgreater{}\%} \FunctionTok{filter}\NormalTok{(}\SpecialCharTok{!}\FunctionTok{is.na}\NormalTok{(Age)) }\SpecialCharTok{\%\textgreater{}\%} \FunctionTok{mean}\NormalTok{(}\FunctionTok{as.numeric}\NormalTok{(Age))}
\end{Highlighting}
\end{Shaded}

\begin{verbatim}
## Warning in mean.default(., as.numeric(Age)): argument is not numeric or logical:
## returning NA
\end{verbatim}

\begin{verbatim}
## [1] NA
\end{verbatim}

\begin{Shaded}
\begin{Highlighting}[]
\CommentTok{\#do mean age of those who are food insecure and mean age of those who are not}
\end{Highlighting}
\end{Shaded}

As educational degree completion rises, so does food security.

White Only, Asian only, and White-Asian identified groups are the most
food secure populations out of other races. Every other race is twice as
food insecure as a proportion of their population.

New Hampshire is a relatively food secure State.

Hispanics are about twice as food insecure as non-Hispanics as a
proportion of their populations

The most food insecure marital status is separated spouses, the least is
married with spouse present

Summary Statistics - Rasch Scores

\begin{Shaded}
\begin{Highlighting}[]
\FunctionTok{summary}\NormalTok{(data\_dec21}\SpecialCharTok{$}\NormalTok{twelveM\_Household\_FI\_rasch)}
\end{Highlighting}
\end{Shaded}

\begin{verbatim}
##    Min. 1st Qu.  Median    Mean 3rd Qu.    Max.    NA's 
##   143.0   172.0   340.0   416.9   543.0  1303.0  114918
\end{verbatim}

\begin{Shaded}
\begin{Highlighting}[]
\NormalTok{data\_dec21 }\SpecialCharTok{\%\textgreater{}\%} \FunctionTok{filter}\NormalTok{(}\SpecialCharTok{!}\FunctionTok{is.na}\NormalTok{(twelveM\_Household\_FI\_rasch)) }\SpecialCharTok{\%\textgreater{}\%} 
\FunctionTok{group\_by}\NormalTok{(Sex) }\SpecialCharTok{\%\textgreater{}\%}
  \FunctionTok{summarise}\NormalTok{(}
    \AttributeTok{n =} \FunctionTok{n}\NormalTok{(),}
    \AttributeTok{Median =} \FunctionTok{median}\NormalTok{(twelveM\_Household\_FI\_rasch),}
    \AttributeTok{mean =} \FunctionTok{mean}\NormalTok{(twelveM\_Household\_FI\_rasch),}
    \AttributeTok{sd =} \FunctionTok{sd}\NormalTok{(twelveM\_Household\_FI\_rasch),}
    \AttributeTok{Min =} \FunctionTok{min}\NormalTok{(twelveM\_Household\_FI\_rasch),}
    \AttributeTok{Max =} \FunctionTok{max}\NormalTok{(twelveM\_Household\_FI\_rasch)}
\NormalTok{  )}
\end{Highlighting}
\end{Shaded}

\begin{verbatim}
## # A tibble: 2 x 7
##   Sex        n Median  mean    sd   Min   Max
##   <fct>  <int>  <dbl> <dbl> <dbl> <dbl> <dbl>
## 1 Male    5958    340  415.  240.   143  1303
## 2 Female  6613    340  418.  243.   143  1303
\end{verbatim}

\begin{Shaded}
\begin{Highlighting}[]
\NormalTok{data\_dec21 }\SpecialCharTok{\%\textgreater{}\%} \FunctionTok{filter}\NormalTok{(}\SpecialCharTok{!}\FunctionTok{is.na}\NormalTok{(twelveM\_Household\_FI\_rasch)) }\SpecialCharTok{\%\textgreater{}\%} 
\FunctionTok{group\_by}\NormalTok{(Race) }\SpecialCharTok{\%\textgreater{}\%}
  \FunctionTok{summarise}\NormalTok{(}
    \AttributeTok{n =} \FunctionTok{n}\NormalTok{(),}
    \AttributeTok{Median =} \FunctionTok{median}\NormalTok{(twelveM\_Household\_FI\_rasch),}
    \AttributeTok{mean =} \FunctionTok{mean}\NormalTok{(twelveM\_Household\_FI\_rasch),}
    \AttributeTok{sd =} \FunctionTok{sd}\NormalTok{(twelveM\_Household\_FI\_rasch),}
    \AttributeTok{Min =} \FunctionTok{min}\NormalTok{(twelveM\_Household\_FI\_rasch),}
    \AttributeTok{Max =} \FunctionTok{max}\NormalTok{(twelveM\_Household\_FI\_rasch)}
\NormalTok{  )}
\end{Highlighting}
\end{Shaded}

\begin{verbatim}
## # A tibble: 9 x 7
##   Race                                     n Median  mean    sd   Min   Max
##   <fct>                                <int>  <dbl> <dbl> <dbl> <dbl> <dbl>
## 1 White Only                            8763    340  413.  239.   143  1303
## 2 Other Mixed Race                       130    414  456.  239.   143   984
## 3 Black Only                            2300    414  435.  246.   143  1216
## 4 American Indian, Alaskan Native Only   338    414  458.  263.   143  1216
## 5 Aisan Only                             587    310  386.  252.   143  1303
## 6 Hawaiian/Pacific Islander Only          99    310  338.  190.   143  1015
## 7 White-Black                            164    340  395.  225.   143  1105
## 8 White-AI                               138    340  435.  272.   143  1105
## 9 White-Asian                             52    340  414.  265.   143  1105
\end{verbatim}

\begin{Shaded}
\begin{Highlighting}[]
\NormalTok{data\_dec21 }\SpecialCharTok{\%\textgreater{}\%} \FunctionTok{filter}\NormalTok{(}\SpecialCharTok{!}\FunctionTok{is.na}\NormalTok{(twelveM\_Household\_FI\_rasch)) }\SpecialCharTok{\%\textgreater{}\%} 
\FunctionTok{group\_by}\NormalTok{(Stabr) }\SpecialCharTok{\%\textgreater{}\%}
  \FunctionTok{summarise}\NormalTok{(}
    \AttributeTok{n =} \FunctionTok{n}\NormalTok{(),}
    \AttributeTok{Median =} \FunctionTok{median}\NormalTok{(twelveM\_Household\_FI\_rasch),}
    \AttributeTok{mean =} \FunctionTok{mean}\NormalTok{(twelveM\_Household\_FI\_rasch),}
    \AttributeTok{sd =} \FunctionTok{sd}\NormalTok{(twelveM\_Household\_FI\_rasch),}
    \AttributeTok{Min =} \FunctionTok{min}\NormalTok{(twelveM\_Household\_FI\_rasch),}
    \AttributeTok{Max =} \FunctionTok{max}\NormalTok{(twelveM\_Household\_FI\_rasch)}
\NormalTok{  )}
\end{Highlighting}
\end{Shaded}

\begin{verbatim}
## # A tibble: 50 x 7
##    Stabr     n Median  mean    sd   Min   Max
##    <fct> <int>  <dbl> <dbl> <dbl> <dbl> <dbl>
##  1 AL      196    423  450.  230.   143  1015
##  2 DE      155    523  507.  236.   143  1105
##  3 DC      180    310  362.  237.   143   984
##  4 FL      476    340  408.  226.   143  1105
##  5 GA      225    340  398.  236.   143  1105
##  6 HI      183    340  385.  233.   143  1015
##  7 ID      241    340  399.  217.   143   898
##  8 IL      321    340  400.  253.   143  1216
##  9 IN      180    523  529.  286.   143  1105
## 10 IA      159    340  380.  215.   143  1105
## # ... with 40 more rows
\end{verbatim}

\begin{Shaded}
\begin{Highlighting}[]
\NormalTok{data\_dec21 }\SpecialCharTok{\%\textgreater{}\%} \FunctionTok{filter}\NormalTok{(}\SpecialCharTok{!}\FunctionTok{is.na}\NormalTok{(twelveM\_Household\_FI\_rasch)) }\SpecialCharTok{\%\textgreater{}\%} 
\FunctionTok{group\_by}\NormalTok{(Hispanic) }\SpecialCharTok{\%\textgreater{}\%}
  \FunctionTok{summarise}\NormalTok{(}
    \AttributeTok{n =} \FunctionTok{n}\NormalTok{(),}
    \AttributeTok{Median =} \FunctionTok{median}\NormalTok{(twelveM\_Household\_FI\_rasch),}
    \AttributeTok{mean =} \FunctionTok{mean}\NormalTok{(twelveM\_Household\_FI\_rasch),}
    \AttributeTok{sd =} \FunctionTok{sd}\NormalTok{(twelveM\_Household\_FI\_rasch),}
    \AttributeTok{Min =} \FunctionTok{min}\NormalTok{(twelveM\_Household\_FI\_rasch),}
    \AttributeTok{Max =} \FunctionTok{max}\NormalTok{(twelveM\_Household\_FI\_rasch)}
\NormalTok{  )}
\end{Highlighting}
\end{Shaded}

\begin{verbatim}
## # A tibble: 2 x 7
##   Hispanic         n Median  mean    sd   Min   Max
##   <fct>        <int>  <dbl> <dbl> <dbl> <dbl> <dbl>
## 1 Hispanic      3064    340  398.  232.   143  1303
## 2 Non-Hispanic  9507    414  423.  244.   143  1303
\end{verbatim}

\begin{Shaded}
\begin{Highlighting}[]
\NormalTok{data\_dec21 }\SpecialCharTok{\%\textgreater{}\%} \FunctionTok{filter}\NormalTok{(}\SpecialCharTok{!}\FunctionTok{is.na}\NormalTok{(twelveM\_Household\_FI\_rasch)) }\SpecialCharTok{\%\textgreater{}\%} 
\FunctionTok{group\_by}\NormalTok{(Marital\_status) }\SpecialCharTok{\%\textgreater{}\%}
  \FunctionTok{summarise}\NormalTok{(}
    \AttributeTok{n =} \FunctionTok{n}\NormalTok{(),}
    \AttributeTok{Median =} \FunctionTok{median}\NormalTok{(twelveM\_Household\_FI\_rasch),}
    \AttributeTok{mean =} \FunctionTok{mean}\NormalTok{(twelveM\_Household\_FI\_rasch),}
    \AttributeTok{sd =} \FunctionTok{sd}\NormalTok{(twelveM\_Household\_FI\_rasch),}
    \AttributeTok{Min =} \FunctionTok{min}\NormalTok{(twelveM\_Household\_FI\_rasch),}
    \AttributeTok{Max =} \FunctionTok{max}\NormalTok{(twelveM\_Household\_FI\_rasch)}
\NormalTok{  )}
\end{Highlighting}
\end{Shaded}

\begin{verbatim}
## # A tibble: 7 x 7
##   Marital_status               n Median  mean    sd   Min   Max
##   <fct>                    <int>  <dbl> <dbl> <dbl> <dbl> <dbl>
## 1 MARRIED - SPOUSE PRESENT  3244    340  387.  225.   143  1303
## 2 MARRIED - SPOUSE ABSENT    168    423  436.  236.   143  1105
## 3 WIDOWED                    648    423  443.  250.   143  1303
## 4 DIVORCED                  1409    423  476.  262.   143  1216
## 5 SEPARATED                  309    423  470.  275.   143  1105
## 6 NEVER MARRIED             3938    414  435.  251.   143  1303
## 7 <NA>                      2855    340  384.  220.   143  1303
\end{verbatim}

\begin{Shaded}
\begin{Highlighting}[]
\NormalTok{data\_dec21 }\SpecialCharTok{\%\textgreater{}\%} \FunctionTok{filter}\NormalTok{(}\SpecialCharTok{!}\FunctionTok{is.na}\NormalTok{(twelveM\_Household\_FI\_rasch)) }\SpecialCharTok{\%\textgreater{}\%} 
\FunctionTok{group\_by}\NormalTok{(High\_Lvl\_Degree) }\SpecialCharTok{\%\textgreater{}\%}
  \FunctionTok{summarise}\NormalTok{(}
    \AttributeTok{n =} \FunctionTok{n}\NormalTok{(),}
    \AttributeTok{Median =} \FunctionTok{median}\NormalTok{(twelveM\_Household\_FI\_rasch),}
    \AttributeTok{mean =} \FunctionTok{mean}\NormalTok{(twelveM\_Household\_FI\_rasch),}
    \AttributeTok{sd =} \FunctionTok{sd}\NormalTok{(twelveM\_Household\_FI\_rasch),}
    \AttributeTok{Min =} \FunctionTok{min}\NormalTok{(twelveM\_Household\_FI\_rasch),}
    \AttributeTok{Max =} \FunctionTok{max}\NormalTok{(twelveM\_Household\_FI\_rasch)}
\NormalTok{  )}
\end{Highlighting}
\end{Shaded}

\begin{verbatim}
## # A tibble: 8 x 7
##   High_Lvl_Degree          n Median  mean    sd   Min   Max
##   <fct>                <int>  <dbl> <dbl> <dbl> <dbl> <dbl>
## 1 Less than H.S or GED  2122    414  432.  245.   143  1303
## 2 H.S or GED            5310    414  426.  246.   143  1303
## 3 Associate's            905    414  428.  250.   143  1303
## 4 Bachelor's            1011    340  417.  247.   143  1303
## 5 Master's               304    340  417.  251.   143  1113
## 6 Professional            30    310  412.  281.   143  1105
## 7 Doctorate               34    481  502.  282.   143  1105
## 8 <NA>                  2855    340  384.  220.   143  1303
\end{verbatim}

\begin{Shaded}
\begin{Highlighting}[]
\NormalTok{data\_dec21 }\SpecialCharTok{\%\textgreater{}\%} \FunctionTok{filter}\NormalTok{(}\SpecialCharTok{!}\FunctionTok{is.na}\NormalTok{(twelveM\_Household\_FI\_rasch)) }\SpecialCharTok{\%\textgreater{}\%} 
\FunctionTok{group\_by}\NormalTok{(Sex, Race) }\SpecialCharTok{\%\textgreater{}\%}
  \FunctionTok{summarise}\NormalTok{(}
    \AttributeTok{n =} \FunctionTok{n}\NormalTok{(),}
    \AttributeTok{Median =} \FunctionTok{median}\NormalTok{(twelveM\_Household\_FI\_rasch),}
    \AttributeTok{mean =} \FunctionTok{mean}\NormalTok{(twelveM\_Household\_FI\_rasch),}
    \AttributeTok{sd =} \FunctionTok{sd}\NormalTok{(twelveM\_Household\_FI\_rasch),}
    \AttributeTok{Min =} \FunctionTok{min}\NormalTok{(twelveM\_Household\_FI\_rasch),}
    \AttributeTok{Max =} \FunctionTok{max}\NormalTok{(twelveM\_Household\_FI\_rasch)}
\NormalTok{  )}
\end{Highlighting}
\end{Shaded}

\begin{verbatim}
## `summarise()` has grouped output by 'Sex'. You can override using the `.groups`
## argument.
\end{verbatim}

\begin{verbatim}
## # A tibble: 18 x 8
## # Groups:   Sex [2]
##    Sex    Race                                  n Median  mean    sd   Min   Max
##    <fct>  <fct>                             <int>  <dbl> <dbl> <dbl> <dbl> <dbl>
##  1 Male   White Only                         4193   340   411.  237.   143  1303
##  2 Male   Other Mixed Race                     64   414   459.  228.   143   984
##  3 Male   Black Only                         1047   423   434.  245.   143  1216
##  4 Male   American Indian, Alaskan Native ~   154   452   470.  252.   143  1105
##  5 Male   Aisan Only                          276   310   383.  247.   143  1303
##  6 Male   Hawaiian/Pacific Islander Only       46   340   334.  165.   143   661
##  7 Male   White-Black                          91   340   402.  235.   143  1105
##  8 Male   White-AI                             56   310   401.  272.   143  1105
##  9 Male   White-Asian                          31   481   473.  281.   143  1105
## 10 Female White Only                         4570   340   415.  240.   143  1303
## 11 Female Other Mixed Race                     66   418.  453.  251.   143   984
## 12 Female Black Only                         1253   414   436.  246.   143  1216
## 13 Female American Indian, Alaskan Native ~   184   377   448.  272.   143  1216
## 14 Female Aisan Only                          311   310   388.  256.   143  1303
## 15 Female Hawaiian/Pacific Islander Only       53   256   342.  211.   143  1015
## 16 Female White-Black                          73   340   387.  214.   143  1105
## 17 Female White-AI                             82   414   458.  271.   143  1105
## 18 Female White-Asian                          21   340   328.  219.   143   898
\end{verbatim}

Within the data documentation there is not an explanation given of the
direction of the food insecurity scale. The means and medians differ but
interpretation is difficult.

\hypertarget{rasch-scores}{%
\subsection{rasch Scores}\label{rasch-scores}}

\begin{Shaded}
\begin{Highlighting}[]
\CommentTok{\#Graphing Food Insecurity Rasch scores by Age/Sex}
\NormalTok{data\_dec21 }\SpecialCharTok{\%\textgreater{}\%} \FunctionTok{filter}\NormalTok{(}\SpecialCharTok{!}\FunctionTok{is.na}\NormalTok{(twelveM\_Household\_FI\_rasch)) }\SpecialCharTok{\%\textgreater{}\%} 
  \FunctionTok{ggplot}\NormalTok{(}\FunctionTok{aes}\NormalTok{(}\AttributeTok{x =}\NormalTok{ Age, }\AttributeTok{y =}\NormalTok{ twelveM\_Household\_FI\_rasch, }\AttributeTok{color =}\NormalTok{ Sex)) }\SpecialCharTok{+}
  \FunctionTok{stat\_summary}\NormalTok{(}\FunctionTok{aes}\NormalTok{(}\AttributeTok{x =}\NormalTok{ Age, }\AttributeTok{y =}\NormalTok{ twelveM\_Household\_FI\_rasch, }\AttributeTok{group =}\NormalTok{ Sex), }\AttributeTok{fun.y=}\NormalTok{mean, }\AttributeTok{geom=}\StringTok{"line"}\NormalTok{, }\AttributeTok{size =}\FloatTok{1.5}\NormalTok{) }\SpecialCharTok{+}
  \FunctionTok{labs}\NormalTok{(}\AttributeTok{y =} \StringTok{"Mean Food Insecurity Rasch Score (12 Month Household)"}\NormalTok{, }\AttributeTok{title =} \StringTok{"Food Insecurity Scores of Sexes across Age"}\NormalTok{, }\AttributeTok{caption =} \StringTok{"Source: Current Population Survey Food Security Supplements"}\NormalTok{) }\SpecialCharTok{+}
  \FunctionTok{theme\_bw}\NormalTok{() }
\end{Highlighting}
\end{Shaded}

\includegraphics{Final_Activity_v3_files/figure-latex/unnamed-chunk-10-1.pdf}

Food Insecurity is roughly parallel between Males and Females across all
ages. The value of the Rasch score rises until mid 50's and then falls
until early 80's for both. There are a couple places though that the
scores diverge, such as at age \textasciitilde50, and early 60's.

Density Plot:

\begin{Shaded}
\begin{Highlighting}[]
\CommentTok{\#Making a new data frame with only associates and bachelor\textquotesingle{}s degrees}
\NormalTok{dgr\_df }\OtherTok{\textless{}{-}}\NormalTok{ data\_dec21 }\SpecialCharTok{\%\textgreater{}\%} \FunctionTok{filter}\NormalTok{(High\_Lvl\_Degree }\SpecialCharTok{==} \StringTok{"Doctorate"} \SpecialCharTok{|}\NormalTok{ High\_Lvl\_Degree }\SpecialCharTok{==} \StringTok{"Less than H.S or GED"}\NormalTok{)}

\CommentTok{\#Density Plot of Food Insecurity rasch scores}
\NormalTok{dgr\_df }\SpecialCharTok{\%\textgreater{}\%} \FunctionTok{filter}\NormalTok{(}\SpecialCharTok{!}\FunctionTok{is.na}\NormalTok{(twelveM\_Household\_FI\_rasch)) }\SpecialCharTok{\%\textgreater{}\%} 
  \FunctionTok{ggplot}\NormalTok{(}\FunctionTok{aes}\NormalTok{(}\AttributeTok{x =}\NormalTok{ twelveM\_Household\_FI\_rasch, }\AttributeTok{fill =}\NormalTok{ High\_Lvl\_Degree)) }\SpecialCharTok{+}
  \FunctionTok{geom\_density}\NormalTok{(}\AttributeTok{alpha =}\NormalTok{ .}\DecValTok{5}\NormalTok{) }\SpecialCharTok{+}
  \FunctionTok{labs}\NormalTok{(}\AttributeTok{x =}\StringTok{"Rasch Score (12 Month Household)"}\NormalTok{,}
       \AttributeTok{title =} \StringTok{"Density plot: Rasch Food Insecurity Scores and Education"}\NormalTok{,}
       \AttributeTok{caption =} \StringTok{"Source: Current Population Survey Food Security Supplements"}\NormalTok{)}
\end{Highlighting}
\end{Shaded}

\includegraphics{Final_Activity_v3_files/figure-latex/unnamed-chunk-11-1.pdf}

Interpretation is difficult given there's not data documentation of the
variable, but among degree attainments, these are the most different.
This manifests in the density plot with less than High School or GED
peaking for low Rasch score values, while Doctorate is much more spread
and contains larger counts of high values. Based on intuition and this
plot alone it would indicate that lower Rasch scores indicate food
insecurity, and higher Rasch scores indicate Food Security.

\hypertarget{merging-data-set-by-county}{%
\subsection{Merging Data set by
County}\label{merging-data-set-by-county}}

Merged\_DataSet is equivalent to data\_dec21 ecept for an additional row
that gives the percentage of the respective county that is in poverty

\begin{Shaded}
\begin{Highlighting}[]
\CommentTok{\#Read in xlsx file}
\NormalTok{poverty\_data\_large }\OtherTok{\textless{}{-}} \FunctionTok{read\_excel}\NormalTok{(}\StringTok{"C:/Users/foste/Desktop/418 final project folder/PovertyEstimates\_new.xlsx"}\NormalTok{)}
\NormalTok{poverty\_data\_county }\OtherTok{\textless{}{-}} \FunctionTok{select}\NormalTok{(poverty\_data\_large,FIPS\_code, PCTPOVALL\_2020)}
\CommentTok{\#rm(poverty\_data\_large)}
\end{Highlighting}
\end{Shaded}

Merging Data sets by inner join

\begin{Shaded}
\begin{Highlighting}[]
\NormalTok{Merged\_DataSet }\OtherTok{\textless{}{-}}\NormalTok{  data\_dec21 }\SpecialCharTok{\%\textgreater{}\%} \FunctionTok{inner\_join}\NormalTok{(poverty\_data\_county,}\AttributeTok{by=}\StringTok{"FIPS\_code"}\NormalTok{)}
\FunctionTok{rm}\NormalTok{(poverty\_data\_large)}
\end{Highlighting}
\end{Shaded}

\hypertarget{fi_ornot-variable}{%
\subsection{FI\_orNot Variable}\label{fi_ornot-variable}}

\begin{Shaded}
\begin{Highlighting}[]
\CommentTok{\#A table }
\NormalTok{Merged\_DataSet }\SpecialCharTok{\%\textgreater{}\%} \FunctionTok{filter}\NormalTok{(}\SpecialCharTok{!}\FunctionTok{is.na}\NormalTok{(FI\_orNot)) }\SpecialCharTok{\%\textgreater{}\%} 
\FunctionTok{tabyl}\NormalTok{(Sex, FI\_orNot) }\SpecialCharTok{\%\textgreater{}\%} 
  \FunctionTok{adorn\_percentages}\NormalTok{(}\StringTok{"row"}\NormalTok{) }\SpecialCharTok{\%\textgreater{}\%} 
  \FunctionTok{adorn\_pct\_formatting}\NormalTok{(}\AttributeTok{digits =} \DecValTok{1}\NormalTok{)}
\end{Highlighting}
\end{Shaded}

\begin{verbatim}
##     Sex High Food Security Food Insecure
##    Male              82.9%         17.1%
##  Female              81.9%         18.1%
\end{verbatim}

\begin{Shaded}
\begin{Highlighting}[]
\CommentTok{\#Creating d5 which holds the 10 counties with the highest percentage of food insecure}
\NormalTok{d4 }\OtherTok{\textless{}{-}}\NormalTok{ Merged\_DataSet }\SpecialCharTok{\%\textgreater{}\%} \FunctionTok{filter}\NormalTok{(}\SpecialCharTok{!}\FunctionTok{is.na}\NormalTok{(FIPS\_code),}\SpecialCharTok{!}\FunctionTok{is.na}\NormalTok{(FI\_orNot)) }\SpecialCharTok{\%\textgreater{}\%} 
  \FunctionTok{group\_by}\NormalTok{(Stabr,FIPS\_code, FI\_orNot) }\SpecialCharTok{\%\textgreater{}\%} 
  \FunctionTok{summarise}\NormalTok{(}\AttributeTok{count =} \FunctionTok{n}\NormalTok{()) }\SpecialCharTok{\%\textgreater{}\%} 
  \FunctionTok{mutate}\NormalTok{(}\AttributeTok{perc\_FI =}\NormalTok{ count}\SpecialCharTok{/}\FunctionTok{sum}\NormalTok{(count))}
\end{Highlighting}
\end{Shaded}

\begin{verbatim}
## `summarise()` has grouped output by 'Stabr', 'FIPS_code'. You can override
## using the `.groups` argument.
\end{verbatim}

\begin{Shaded}
\begin{Highlighting}[]
\NormalTok{d5 }\OtherTok{\textless{}{-}}\NormalTok{ d4 }\SpecialCharTok{\%\textgreater{}\%} \FunctionTok{filter}\NormalTok{(FI\_orNot }\SpecialCharTok{==} \StringTok{"Food Insecure"}\NormalTok{) }\SpecialCharTok{\%\textgreater{}\%} \FunctionTok{arrange}\NormalTok{(}\FunctionTok{desc}\NormalTok{(perc\_FI))}

\NormalTok{d5 }\OtherTok{\textless{}{-}} \FunctionTok{head}\NormalTok{(d5, }\DecValTok{10}\NormalTok{)}

\CommentTok{\#Top ten food insecure counties I have data on}
\NormalTok{d5 }\SpecialCharTok{\%\textgreater{}\%} \FunctionTok{filter}\NormalTok{(FI\_orNot }\SpecialCharTok{==} \StringTok{"Food Insecure"}\NormalTok{) }\SpecialCharTok{\%\textgreater{}\%} 
  \FunctionTok{ggplot}\NormalTok{(}\FunctionTok{aes}\NormalTok{(}\AttributeTok{x =} \FunctionTok{fct\_reorder}\NormalTok{(FIPS\_code, perc\_FI, }\AttributeTok{.desc =} \ConstantTok{TRUE}\NormalTok{), }\AttributeTok{y =}\NormalTok{ perc\_FI}\SpecialCharTok{*}\DecValTok{100}\NormalTok{)) }\SpecialCharTok{+}
  \FunctionTok{geom\_bar}\NormalTok{(}\AttributeTok{stat =} \StringTok{"identity"}\NormalTok{, }\AttributeTok{width =}\NormalTok{ .}\DecValTok{7}\NormalTok{, }\AttributeTok{fill =} \StringTok{"black"}\NormalTok{) }\SpecialCharTok{+}
  \FunctionTok{theme}\NormalTok{(}\AttributeTok{axis.text.x =} \FunctionTok{element\_text}\NormalTok{(}\AttributeTok{angle=}\DecValTok{45}\NormalTok{, }\AttributeTok{vjust=}\DecValTok{1}\NormalTok{, }\AttributeTok{hjust=}\DecValTok{1}\NormalTok{)) }\SpecialCharTok{+}
  \FunctionTok{labs}\NormalTok{(}\AttributeTok{title =} \StringTok{"Top Ten Food Insecure Counties"}\NormalTok{,}\AttributeTok{y =} \StringTok{"Percentage Food Insecure of County"}\NormalTok{,}
       \AttributeTok{x =} \StringTok{"County FIPS Code "}\NormalTok{,}
       \AttributeTok{caption =} \StringTok{"Source: Current Population Survey Food Security Supplements"}\NormalTok{) }\SpecialCharTok{+}
  \FunctionTok{geom\_text}\NormalTok{(}\FunctionTok{aes}\NormalTok{(}\AttributeTok{label =}\NormalTok{ Stabr), }\AttributeTok{nudge\_y =} \SpecialCharTok{{-}}\DecValTok{3}\NormalTok{, }\AttributeTok{color =} \StringTok{"white"}\NormalTok{)}
\end{Highlighting}
\end{Shaded}

\includegraphics{Final_Activity_v3_files/figure-latex/unnamed-chunk-14-1.pdf}

This bar graph shows the top 10 counties in the data for food
insecurity, specifically using the FI\_orNot variable, and looking at
the percentage of that county in the data that is food insecure. One can
look up more information on these counties using the FIPS code shown on
the graph. These FIPS codes are standardized across the U.S. The
abbreviation of the respective States was also added to the plot. This
plot shows two counties from Texas in the top 10, as well as two
counties from Californian.

RACE:

\begin{Shaded}
\begin{Highlighting}[]
\CommentTok{\# returns a data\_frame that has the percentage of each race that is food insecure or not}
\NormalTok{d2 }\OtherTok{\textless{}{-}}\NormalTok{ Merged\_DataSet }\SpecialCharTok{\%\textgreater{}\%} \FunctionTok{filter}\NormalTok{(}\SpecialCharTok{!}\FunctionTok{is.na}\NormalTok{(Race),}\SpecialCharTok{!}\FunctionTok{is.na}\NormalTok{(FI\_orNot)) }\SpecialCharTok{\%\textgreater{}\%} 
  \FunctionTok{group\_by}\NormalTok{(Race, FI\_orNot) }\SpecialCharTok{\%\textgreater{}\%} 
  \FunctionTok{summarise}\NormalTok{(}\AttributeTok{count =} \FunctionTok{n}\NormalTok{()) }\SpecialCharTok{\%\textgreater{}\%} 
  \FunctionTok{mutate}\NormalTok{(}\AttributeTok{perc\_FIofRace =}\NormalTok{ count}\SpecialCharTok{/}\FunctionTok{sum}\NormalTok{(count))}
\end{Highlighting}
\end{Shaded}

\begin{verbatim}
## `summarise()` has grouped output by 'Race'. You can override using the
## `.groups` argument.
\end{verbatim}

\begin{Shaded}
\begin{Highlighting}[]
\CommentTok{\#ggplot for percentage of each race that is food insecure}
\CommentTok{\#d2 \%\textgreater{}\% filter(FI\_orNot == "Food Insecure") \%\textgreater{}\% }
\CommentTok{\#  ggplot(aes(x = fct\_reorder(Race, perc\_FIofRace, .desc = TRUE), y = perc\_FIofRace*100)) +}
\CommentTok{\#  geom\_bar(stat = "identity", width = .7) +}
\CommentTok{\#  theme(axis.text.x = element\_text(angle=70, vjust=1, hjust=1)) +}
\CommentTok{\#  labs(title = "Food Insecurity and Race",y = "Percentage Food Insecure ", }
 \CommentTok{\#      caption = "Source: Current Population Survey Food Security Supplements") }
  

\CommentTok{\#now find the racial makeup of those counties, d6 holds the racial makeup for all counties with data}
\NormalTok{d6 }\OtherTok{\textless{}{-}}\NormalTok{ Merged\_DataSet }\SpecialCharTok{\%\textgreater{}\%} \FunctionTok{filter}\NormalTok{(}\SpecialCharTok{!}\FunctionTok{is.na}\NormalTok{(FIPS\_code),}\SpecialCharTok{!}\FunctionTok{is.na}\NormalTok{(FI\_orNot), }\SpecialCharTok{!}\FunctionTok{is.na}\NormalTok{(Race)) }\SpecialCharTok{\%\textgreater{}\%} 
  \FunctionTok{group\_by}\NormalTok{(Stabr,FIPS\_code, Race) }\SpecialCharTok{\%\textgreater{}\%} 
  \FunctionTok{summarise}\NormalTok{(}\AttributeTok{count =} \FunctionTok{n}\NormalTok{()) }\SpecialCharTok{\%\textgreater{}\%} 
  \FunctionTok{mutate}\NormalTok{(}\AttributeTok{perc\_RaceCounty =}\NormalTok{ count}\SpecialCharTok{/}\FunctionTok{sum}\NormalTok{(count))}
\end{Highlighting}
\end{Shaded}

\begin{verbatim}
## `summarise()` has grouped output by 'Stabr', 'FIPS_code'. You can override
## using the `.groups` argument.
\end{verbatim}

\begin{Shaded}
\begin{Highlighting}[]
\CommentTok{\#Merging data sets to create}
\NormalTok{Merged\_FICounty\_Race }\OtherTok{\textless{}{-}}\NormalTok{ d5 }\SpecialCharTok{\%\textgreater{}\%} \FunctionTok{inner\_join}\NormalTok{(d6,}\AttributeTok{by=}\StringTok{"FIPS\_code"}\NormalTok{)}


\CommentTok{\#Racial Makeup of 10 most food insecure counties by percentage}
\NormalTok{Merged\_FICounty\_Race }\SpecialCharTok{\%\textgreater{}\%}
  \FunctionTok{ggplot}\NormalTok{(}\FunctionTok{aes}\NormalTok{(}\AttributeTok{x =} \FunctionTok{fct\_reorder}\NormalTok{(FIPS\_code, perc\_FI, }\AttributeTok{.desc =} \ConstantTok{TRUE}\NormalTok{), }\AttributeTok{y =}\NormalTok{ perc\_RaceCounty}\SpecialCharTok{*}\DecValTok{100}\NormalTok{, }\AttributeTok{fill =}\NormalTok{ Race)) }\SpecialCharTok{+}
  \FunctionTok{geom\_bar}\NormalTok{(}\AttributeTok{stat =} \StringTok{"identity"}\NormalTok{, }\AttributeTok{width =}\NormalTok{ .}\DecValTok{7}\NormalTok{, }\AttributeTok{color =} \StringTok{"black"}\NormalTok{) }\SpecialCharTok{+}
  \FunctionTok{labs}\NormalTok{(}\AttributeTok{title =} \StringTok{"Racial Demographics of 10 Most Food Insecure Counties"}\NormalTok{,}
       \AttributeTok{x =} \StringTok{"County FIPS Code in Descending order of Food Insecurity"}\NormalTok{,}
       \AttributeTok{y =} \StringTok{"Percentage Food Insecure "}\NormalTok{,}
       \AttributeTok{caption =} \StringTok{"Source: Current Population Survey Food Security Supplements"}\NormalTok{) }\SpecialCharTok{+}
  \FunctionTok{scale\_fill\_colorblind}\NormalTok{() }\SpecialCharTok{+}
  \FunctionTok{theme}\NormalTok{(}\AttributeTok{axis.text.x =} \FunctionTok{element\_text}\NormalTok{(}\AttributeTok{angle=}\DecValTok{45}\NormalTok{, }\AttributeTok{vjust=}\DecValTok{1}\NormalTok{, }\AttributeTok{hjust=}\DecValTok{1}\NormalTok{)) }
\end{Highlighting}
\end{Shaded}

\includegraphics{Final_Activity_v3_files/figure-latex/unnamed-chunk-15-1.pdf}

The top 10 food insecure counties in the data were taken then and put in
the exact same order as previously listed, as in descending order from
most food insecure as a percentage of the population to least. We now
look at the data within these food insecure counties to look at racial
demographics. In a couple of the counties there is a large Black
population, which would be consistent with earlier data exploration
findings of the association between food insecurity and race.

Looking at the data directly the confidence of these percentages being
accurate to the real county demographics is very questionable. In 39103
there was only 7 people surveyed, all of which were white. A very small
sample will not give us high confidence in neither the food insecurity
proportion nor the racial demographic spread. Unique county codes had to
be created from code and do not exist originally in the Current
Population Food Security Supplements, and so I determine that the
survery was most likley not intended for county level descriptions.

This racial makeup data does not take into account Hispanic status:

HISPANIC

\begin{Shaded}
\begin{Highlighting}[]
\CommentTok{\#creating a data frame that has the percentage hispanic in each in all counties with data}
\NormalTok{d7 }\OtherTok{\textless{}{-}}\NormalTok{ Merged\_DataSet }\SpecialCharTok{\%\textgreater{}\%} \FunctionTok{filter}\NormalTok{(}\SpecialCharTok{!}\FunctionTok{is.na}\NormalTok{(FIPS\_code),}\SpecialCharTok{!}\FunctionTok{is.na}\NormalTok{(FI\_orNot), }\SpecialCharTok{!}\FunctionTok{is.na}\NormalTok{(Hispanic)) }\SpecialCharTok{\%\textgreater{}\%} 
  \FunctionTok{group\_by}\NormalTok{(Stabr,FIPS\_code, Hispanic) }\SpecialCharTok{\%\textgreater{}\%} 
  \FunctionTok{summarise}\NormalTok{(}\AttributeTok{count =} \FunctionTok{n}\NormalTok{()) }\SpecialCharTok{\%\textgreater{}\%} 
  \FunctionTok{mutate}\NormalTok{(}\AttributeTok{perc\_HispanicCounty =}\NormalTok{ count}\SpecialCharTok{/}\FunctionTok{sum}\NormalTok{(count))}
\end{Highlighting}
\end{Shaded}

\begin{verbatim}
## `summarise()` has grouped output by 'Stabr', 'FIPS_code'. You can override
## using the `.groups` argument.
\end{verbatim}

\begin{Shaded}
\begin{Highlighting}[]
\NormalTok{d8 }\OtherTok{\textless{}{-}}\NormalTok{ d7 }\SpecialCharTok{\%\textgreater{}\%} \FunctionTok{filter}\NormalTok{(Hispanic }\SpecialCharTok{==} \StringTok{"Hispanic"}\NormalTok{)}

\CommentTok{\#Merging to create a new data from top 10 most food insecure counties with }
\NormalTok{Merged\_FICounty\_Hispanic\_Race }\OtherTok{\textless{}{-}}\NormalTok{ d5 }\SpecialCharTok{\%\textgreater{}\%} \FunctionTok{inner\_join}\NormalTok{(d8,}\AttributeTok{by=}\StringTok{"FIPS\_code"}\NormalTok{)}

\CommentTok{\#PIE chart for Hispanic ID proportion of county}
\NormalTok{d7 }\SpecialCharTok{\%\textgreater{}\%}  \FunctionTok{filter}\NormalTok{(FIPS\_code }\SpecialCharTok{==} \StringTok{"06053"}\NormalTok{) }\SpecialCharTok{\%\textgreater{}\%} \FunctionTok{ggplot}\NormalTok{(}\FunctionTok{aes}\NormalTok{(}\AttributeTok{x =} \StringTok{""}\NormalTok{, }\AttributeTok{y =}\NormalTok{ perc\_HispanicCounty, }\AttributeTok{fill =}\NormalTok{ Hispanic)) }\SpecialCharTok{+}
  \FunctionTok{geom\_bar}\NormalTok{(}\AttributeTok{stat =} \StringTok{"identity"}\NormalTok{, }\AttributeTok{width =} \DecValTok{1}\NormalTok{) }\SpecialCharTok{+}
  \FunctionTok{coord\_polar}\NormalTok{(}\StringTok{"y"}\NormalTok{, }\AttributeTok{start=}\DecValTok{0}\NormalTok{) }\SpecialCharTok{+} 
  \FunctionTok{geom\_text}\NormalTok{(}\FunctionTok{aes}\NormalTok{(}\AttributeTok{label =} \FunctionTok{paste0}\NormalTok{(}\FunctionTok{round}\NormalTok{(perc\_HispanicCounty}\SpecialCharTok{*}\DecValTok{100}\NormalTok{, }\DecValTok{1}\NormalTok{), }\StringTok{"\%"}\NormalTok{)), }\AttributeTok{position =} \FunctionTok{position\_stack}\NormalTok{(}\AttributeTok{vjust=}\FloatTok{0.5}\NormalTok{), }\AttributeTok{color =} \StringTok{"white"}\NormalTok{) }\SpecialCharTok{+} 
  \FunctionTok{labs}\NormalTok{(}\AttributeTok{x =}\ConstantTok{NULL}\NormalTok{, }\AttributeTok{y =} \ConstantTok{NULL}\NormalTok{, }\AttributeTok{fill =} \ConstantTok{NULL}\NormalTok{,}
    \AttributeTok{title =} \StringTok{"Percent Hispanic }\SpecialCharTok{\textbackslash{}n}\StringTok{ of County (06053)"}\NormalTok{,}
    \AttributeTok{caption =} \StringTok{"Source: Current Population Survey Food Security Supplements"}\NormalTok{) }\SpecialCharTok{+}
  \FunctionTok{theme\_few}\NormalTok{() }\SpecialCharTok{+}
  \FunctionTok{scale\_fill\_colorblind}\NormalTok{() }\SpecialCharTok{+}
  \FunctionTok{theme}\NormalTok{(}\AttributeTok{plot.title =} \FunctionTok{element\_text}\NormalTok{(}\AttributeTok{hjust =} \FloatTok{0.5}\NormalTok{, }\AttributeTok{vjust =} \FloatTok{0.5}\NormalTok{), }\AttributeTok{plot.caption =} \FunctionTok{element\_text}\NormalTok{(}\AttributeTok{hjust =} \FloatTok{0.5}\NormalTok{, }\AttributeTok{vjust =} \FloatTok{0.5}\NormalTok{))}
\end{Highlighting}
\end{Shaded}

\includegraphics{Final_Activity_v3_files/figure-latex/unnamed-chunk-16-1.pdf}
Hispanic identification is not coded along with race, as the data
documentation explains that any race can be Hispanic. But looking at the
racial categories drawn in the data, it is missing some categories that
live in the minds of the average American. Hispanic ID is associated
with some of these categories.

Looking at the county 06053 from the top ten food insecure counties,
which is Monterey County, CA, the population is 57\% Hispanic. THis
county also had a low percentage of non-white residents compared to some
other counties in the top ten.

RACE AND HISPANIC

\begin{Shaded}
\begin{Highlighting}[]
\CommentTok{\#merging data to have both race and Hispanic percentages in a datafile, i}
\NormalTok{merged\_hisp\_race\_data }\OtherTok{\textless{}{-}}\NormalTok{ Merged\_FICounty\_Race }\SpecialCharTok{\%\textgreater{}\%} \FunctionTok{full\_join}\NormalTok{(Merged\_FICounty\_Hispanic\_Race,}\AttributeTok{by=}\StringTok{"FIPS\_code"}\NormalTok{)}

\CommentTok{\#Racial and Hispanic Makeup of 10 most food insecure counties by percentage}
\NormalTok{merged\_hisp\_race\_data }\SpecialCharTok{\%\textgreater{}\%}
  \FunctionTok{ggplot}\NormalTok{(}\FunctionTok{aes}\NormalTok{(}\AttributeTok{x =} \FunctionTok{fct\_reorder}\NormalTok{(FIPS\_code, perc\_FI.x, }\AttributeTok{.desc =} \ConstantTok{TRUE}\NormalTok{),}\AttributeTok{y =}\NormalTok{ perc\_RaceCounty}\SpecialCharTok{*}\DecValTok{100}\NormalTok{, }\AttributeTok{fill =}\NormalTok{ Race)) }\SpecialCharTok{+}
  \FunctionTok{geom\_col}\NormalTok{(}\AttributeTok{stat =} \StringTok{"identity"}\NormalTok{, }\AttributeTok{width =}\NormalTok{ .}\DecValTok{7}\NormalTok{, }\AttributeTok{color =} \StringTok{"black"}\NormalTok{) }\SpecialCharTok{+}
  \FunctionTok{labs}\NormalTok{(}\AttributeTok{title =} \StringTok{"Racial and Hispanic Proportion (Separately) of}\SpecialCharTok{\textbackslash{}n}\StringTok{ 10 Most Food Insecure Counties"}\NormalTok{,}
       \AttributeTok{x =} \StringTok{"County FIPS Code in Descending order of Food Insecurity"}\NormalTok{,}
       \AttributeTok{y =} \StringTok{"Percentage of County Population"}\NormalTok{,}
       \AttributeTok{caption =} \StringTok{"Source: Current Population Survey Food Security Supplements"}\NormalTok{) }\SpecialCharTok{+}
  \FunctionTok{scale\_fill\_colorblind}\NormalTok{() }\SpecialCharTok{+} 
  \FunctionTok{geom\_col}\NormalTok{(}\FunctionTok{aes}\NormalTok{(}\AttributeTok{y =}\NormalTok{ perc\_HispanicCounty}\SpecialCharTok{*}\DecValTok{100}\NormalTok{, }\AttributeTok{fill =} \StringTok{"Hispanic"}\NormalTok{), }
         \AttributeTok{width =} \FloatTok{0.3}\NormalTok{,}
         \AttributeTok{position =} \FunctionTok{position\_nudge}\NormalTok{(}\AttributeTok{x =} \FloatTok{0.3}\NormalTok{),}
         \AttributeTok{color =} \StringTok{"black"}\NormalTok{) }\SpecialCharTok{+}
   \FunctionTok{theme\_bw}\NormalTok{() }\SpecialCharTok{+}
  \FunctionTok{theme}\NormalTok{(}\AttributeTok{axis.text.x =} \FunctionTok{element\_text}\NormalTok{(}\AttributeTok{angle=}\DecValTok{45}\NormalTok{, }\AttributeTok{vjust=}\DecValTok{1}\NormalTok{, }\AttributeTok{hjust=}\DecValTok{1}\NormalTok{)) }
\end{Highlighting}
\end{Shaded}

\includegraphics{Final_Activity_v3_files/figure-latex/unnamed-chunk-17-1.pdf}

We now overlay the percentage of the county that is Hispanic onto the
ggplot containing racial demographic proportions. This percentage
Hispanic is a separate variable to the percentage of each race variable.
If this plot is not read correctly then one could have incorrect
interpretations about what percentage of ``White Only'' persons are
Hispanic vs.~percentage of ``Black Only.'' The second bar indicating
Hispanic ID is also not meant to be stacked on top of the other Race
levels (this would indicate that only the white coded population is
Hispanic which is not true for every county).

If more coding was done, the height of each Hispanic bar could be raised
or lowered depending on the share of each coded race (for a specific
county) that is Hispanic. Data is available in the CPS to do so.

In counties such as 48135 and 06053 there is a large percentage of
Hispanic people, while county 39103 has a completely White and
non-Hispanic survey sample. The survey sample is very small, but not far
from the demographics listed on the county's Wikipedia:
\url{https://en.wikipedia.org/wiki/Medina_County,_Ohio}.

Based on the visuals of this plot, Hispanic identity most likely is a
piece of the puzzle left out by CPS Racial Demographic descriptions of
food insecurity in Counties. Those looking to demonstrate disparities
caused by historical or current discrimination should take this into
account.

\hypertarget{pivoting-wider}{%
\subsection{Pivoting Wider}\label{pivoting-wider}}

\begin{Shaded}
\begin{Highlighting}[]
\CommentTok{\# returns a data\_frame that has the percentage of each Sex that is food insecure or not}
\NormalTok{d3 }\OtherTok{\textless{}{-}}\NormalTok{ Merged\_DataSet }\SpecialCharTok{\%\textgreater{}\%} \FunctionTok{filter}\NormalTok{(}\SpecialCharTok{!}\FunctionTok{is.na}\NormalTok{(Sex),}\SpecialCharTok{!}\FunctionTok{is.na}\NormalTok{(FI\_orNot)) }\SpecialCharTok{\%\textgreater{}\%} 
  \FunctionTok{group\_by}\NormalTok{(Sex, FI\_orNot) }\SpecialCharTok{\%\textgreater{}\%} 
  \FunctionTok{summarise}\NormalTok{(}\AttributeTok{count =} \FunctionTok{n}\NormalTok{()) }\SpecialCharTok{\%\textgreater{}\%} 
  \FunctionTok{mutate}\NormalTok{(}\AttributeTok{perc\_FIofSex =}\NormalTok{ count}\SpecialCharTok{/}\FunctionTok{sum}\NormalTok{(count))}
\end{Highlighting}
\end{Shaded}

\begin{verbatim}
## `summarise()` has grouped output by 'Sex'. You can override using the `.groups`
## argument.
\end{verbatim}

\begin{Shaded}
\begin{Highlighting}[]
\CommentTok{\#pivot the data frame into a wide format}
\NormalTok{d3\_wider }\OtherTok{\textless{}{-}}\NormalTok{ d3 }\SpecialCharTok{\%\textgreater{}\%} 
  \FunctionTok{pivot\_wider}\NormalTok{(}
    \AttributeTok{names\_from =}\NormalTok{ FI\_orNot, }
    \AttributeTok{values\_from =}\NormalTok{ count}
\NormalTok{  )}

\FunctionTok{glimpse}\NormalTok{(d3\_wider)}
\end{Highlighting}
\end{Shaded}

\begin{verbatim}
## Rows: 4
## Columns: 4
## Groups: Sex [2]
## $ Sex                  <fct> Male, Male, Female, Female
## $ perc_FIofSex         <dbl> 0.8289700, 0.1710300, 0.8194945, 0.1805055
## $ `High Food Security` <int> 28878, NA, 30023, NA
## $ `Food Insecure`      <int> NA, 5958, NA, 6613
\end{verbatim}

\end{document}
